\documentclass[11pt]{article}
\usepackage[utf8]{inputenc}
\usepackage{amsmath}
\usepackage{amsfonts}
\usepackage{amssymb}
\usepackage{geometry}
\usepackage{enumitem}
\usepackage{graphicx}
\usepackage{tikz}
\usepackage{tikz-cd}
\usepackage{pgfplots}
\usepackage{amsthm}
\usepackage{mathtools}

\geometry{margin=1in}

\theoremstyle{definition}
\newtheorem{definition}{Definition}[section]
\newtheorem{theorem}{Theorem}[section]
\newtheorem{lemma}{Lemma}[section]
\newtheorem{corollary}{Corollary}[section]
\newtheorem{example}{Example}[section]
\newtheorem{proposition}{Proposition}[section]

\title{Category Theory Summary}
\author{Mathematical Notes}
\date{\today}

\begin{document}

\maketitle

\tableofcontents
\newpage

\section{Basic Definitions}

\subsection{Categories}
\begin{definition}
A \textbf{category} $\mathcal{C}$ consists of:
\begin{itemize}
    \item A collection of \textbf{objects} $\text{Ob}(\mathcal{C})$
    \item For each pair of objects $A, B$, a set $\text{Hom}(A,B)$ of \textbf{morphisms} (or arrows)
    \item For each object $A$, an \textbf{identity morphism} $1_A: A \to A$
    \item A \textbf{composition} operation $\circ: \text{Hom}(B,C) \times \text{Hom}(A,B) \to \text{Hom}(A,C)$
\end{itemize}
satisfying:
\begin{itemize}
    \item \textbf{Associativity}: $(f \circ g) \circ h = f \circ (g \circ h)$
    \item \textbf{Identity}: $f \circ 1_A = f = 1_B \circ f$ for $f: A \to B$
\end{itemize}
\end{definition}

\subsection{Morphisms}
\begin{definition}
A morphism $f: A \to B$ is:
\begin{itemize}
    \item \textbf{Monic} (monomorphism) if $f \circ g = f \circ h$ implies $g = h$
    \item \textbf{Epic} (epimorphism) if $g \circ f = h \circ f$ implies $g = h$
    \item \textbf{Iso} (isomorphism) if there exists $g: B \to A$ such that $f \circ g = 1_B$ and $g \circ f = 1_A$
\end{itemize}
\end{definition}

\subsection{Examples of Categories}
\begin{example}
\begin{itemize}
    \item \textbf{Set}: Objects are sets, morphisms are functions
    \item \textbf{Grp}: Objects are groups, morphisms are group homomorphisms
    \item \textbf{Top}: Objects are topological spaces, morphisms are continuous maps
    \item \textbf{Vect}$_k$: Objects are vector spaces over field $k$, morphisms are linear maps
    \item \textbf{Pos}: Objects are partially ordered sets, morphisms are order-preserving maps
\end{itemize}
\end{example}

\section{Functors}

\subsection{Definition}
\begin{definition}
A \textbf{functor} $F: \mathcal{C} \to \mathcal{D}$ consists of:
\begin{itemize}
    \item A function $F: \text{Ob}(\mathcal{C}) \to \text{Ob}(\mathcal{D})$
    \item For each $f: A \to B$ in $\mathcal{C}$, a morphism $F(f): F(A) \to F(B)$ in $\mathcal{D}$
\end{itemize}
satisfying:
\begin{itemize}
    \item $F(1_A) = 1_{F(A)}$
    \item $F(f \circ g) = F(f) \circ F(g)$
\end{itemize}
\end{definition}

\subsection{Types of Functors}
\begin{definition}
A functor $F: \mathcal{C} \to \mathcal{D}$ is:
\begin{itemize}
    \item \textbf{Covariant} if $F(f: A \to B) = F(f): F(A) \to F(B)$
    \item \textbf{Contravariant} if $F(f: A \to B) = F(f): F(B) \to F(A)$
    \item \textbf{Full} if $\text{Hom}(A,B) \to \text{Hom}(F(A),F(B))$ is surjective
    \item \textbf{Faithful} if $\text{Hom}(A,B) \to \text{Hom}(F(A),F(B))$ is injective
\end{itemize}
\end{definition}

\subsection{Examples of Functors}
\begin{example}
\begin{itemize}
    \item \textbf{Forgetful functor}: $U: \text{Grp} \to \text{Set}$ sends groups to their underlying sets
    \item \textbf{Free functor}: $F: \text{Set} \to \text{Grp}$ sends sets to free groups
    \item \textbf{Hom functor}: $\text{Hom}(A,-): \mathcal{C} \to \text{Set}$ sends $B$ to $\text{Hom}(A,B)$
    \item \textbf{Power set functor}: $P: \text{Set} \to \text{Set}$ sends sets to their power sets
\end{itemize}
\end{example}

\section{Natural Transformations}

\subsection{Definition}
\begin{definition}
A \textbf{natural transformation} $\eta: F \Rightarrow G$ between functors $F, G: \mathcal{C} \to \mathcal{D}$ consists of:
\begin{itemize}
    \item For each object $A$ in $\mathcal{C}$, a morphism $\eta_A: F(A) \to G(A)$
\end{itemize}
such that for any morphism $f: A \to B$, the following diagram commutes:
$$\begin{tikzcd}
F(A) \arrow[r, "F(f)"] \arrow[d, "\eta_A"] & F(B) \arrow[d, "\eta_B"] \\
G(A) \arrow[r, "G(f)"] & G(B)
\end{tikzcd}$$
\end{definition}

\subsection{Natural Isomorphism}
\begin{definition}
A natural transformation $\eta: F \Rightarrow G$ is a \textbf{natural isomorphism} if each $\eta_A$ is an isomorphism.
\end{definition}

\section{Limits and Colimits}

\subsection{Cones and Cocones}
\begin{definition}
Given a diagram $D: \mathcal{J} \to \mathcal{C}$, a \textbf{cone} over $D$ consists of:
\begin{itemize}
    \item An object $C$ in $\mathcal{C}$
    \item For each object $j$ in $\mathcal{J}$, a morphism $c_j: C \to D(j)$
\end{itemize}
such that for any morphism $f: j \to j'$ in $\mathcal{J}$, we have $D(f) \circ c_j = c_{j'}$.
\end{definition}

\begin{definition}
A \textbf{limit} of a diagram $D: \mathcal{J} \to \mathcal{C}$ is a cone $(L, \lambda)$ that is universal: for any other cone $(C, c)$, there exists a unique morphism $u: C \to L$ such that $\lambda_j \circ u = c_j$ for all $j$.
\end{definition}

\subsection{Colimits}
\begin{definition}
A \textbf{colimit} of a diagram $D: \mathcal{J} \to \mathcal{C}$ is a cocone $(C, c)$ that is universal: for any other cocone $(L, \lambda)$, there exists a unique morphism $u: C \to L$ such that $u \circ c_j = \lambda_j$ for all $j$.
\end{definition}

\subsection{Specific Limits and Colimits}
\begin{definition}
\begin{itemize}
    \item \textbf{Product}: Limit of a discrete diagram
    \item \textbf{Coproduct}: Colimit of a discrete diagram
    \item \textbf{Equalizer}: Limit of a parallel pair
    \item \textbf{Coequalizer}: Colimit of a parallel pair
    \item \textbf{Pullback}: Limit of a cospan
    \item \textbf{Pushout}: Colimit of a span
\end{itemize}
\end{definition}

\section{Adjoint Functors}

\subsection{Definition}
\begin{definition}
Functors $F: \mathcal{C} \to \mathcal{D}$ and $G: \mathcal{D} \to \mathcal{C}$ are \textbf{adjoint} (written $F \dashv G$) if there exists a natural isomorphism:
$$\text{Hom}_{\mathcal{D}}(F(A), B) \cong \text{Hom}_{\mathcal{C}}(A, G(B))$$
\end{definition}

\subsection{Unit and Counit}
\begin{definition}
For adjoint functors $F \dashv G$:
\begin{itemize}
    \item The \textbf{unit} is $\eta: 1_{\mathcal{C}} \Rightarrow G \circ F$
    \item The \textbf{counit} is $\epsilon: F \circ G \Rightarrow 1_{\mathcal{D}}$
\end{itemize}
satisfying the triangle identities:
\begin{itemize}
    \item $\epsilon_{F(A)} \circ F(\eta_A) = 1_{F(A)}$
    \item $G(\epsilon_B) \circ \eta_{G(B)} = 1_{G(B)}$
\end{itemize}
\end{definition}

\subsection{Examples of Adjoints}
\begin{example}
\begin{itemize}
    \item \textbf{Free-Forgetful}: $F \dashv U: \text{Grp} \to \text{Set}$
    \item \textbf{Tensor-Hom}: $-\otimes A \dashv \text{Hom}(A,-)$ in vector spaces
    \item \textbf{Product-Exponential}: $A \times - \dashv (-)^A$ in cartesian closed categories
\end{itemize}
\end{example}

\section{Monads}

\subsection{Definition}
\begin{definition}
A \textbf{monad} on a category $\mathcal{C}$ is a triple $(T, \eta, \mu)$ where:
\begin{itemize}
    \item $T: \mathcal{C} \to \mathcal{C}$ is a functor
    \item $\eta: 1_{\mathcal{C}} \Rightarrow T$ (unit)
    \item $\mu: T^2 \Rightarrow T$ (multiplication)
\end{itemize}
satisfying:
\begin{itemize}
    \item $\mu \circ T\mu = \mu \circ \mu T$ (associativity)
    \item $\mu \circ T\eta = \mu \circ \eta T = 1_T$ (unit laws)
\end{itemize}
\end{definition}

\subsection{Monad Algebras}
\begin{definition}
A \textbf{T-algebra} for a monad $(T, \eta, \mu)$ is a pair $(A, \alpha)$ where:
\begin{itemize}
    \item $A$ is an object in $\mathcal{C}$
    \item $\alpha: T(A) \to A$ is a morphism
\end{itemize}
satisfying:
\begin{itemize}
    \item $\alpha \circ \eta_A = 1_A$
    \item $\alpha \circ T(\alpha) = \alpha \circ \mu_A$
\end{itemize}
\end{definition}

\subsection{Examples of Monads}
\begin{example}
\begin{itemize}
    \item \textbf{List monad}: $T(A) = \text{List}(A)$
    \item \textbf{Maybe monad}: $T(A) = A \cup \{\bot\}$
    \item \textbf{State monad}: $T(A) = S \to (A \times S)$
    \item \textbf{Continuation monad}: $T(A) = (A \to R) \to R$
\end{itemize}
\end{example}

\section{Yoneda Lemma}

\subsection{Presheaves}
\begin{definition}
A \textbf{presheaf} on a category $\mathcal{C}$ is a functor $\mathcal{C}^{\text{op}} \to \text{Set}$.
\end{definition}

\subsection{Yoneda Embedding}
\begin{definition}
The \textbf{Yoneda embedding} is the functor $Y: \mathcal{C} \to [\mathcal{C}^{\text{op}}, \text{Set}]$ defined by:
$$Y(A) = \text{Hom}(-, A)$$
\end{definition}

\subsection{Yoneda Lemma}
\begin{theorem}[Yoneda Lemma]
For any presheaf $F: \mathcal{C}^{\text{op}} \to \text{Set}$ and object $A$ in $\mathcal{C}$:
$$[\mathcal{C}^{\text{op}}, \text{Set}](\text{Hom}(-, A), F) \cong F(A)$$
\end{theorem}

\subsection{Corollary}
\begin{corollary}
The Yoneda embedding is full and faithful.
\end{corollary}

\section{Topoi}

\subsection{Definition}
\begin{definition}
An \textbf{elementary topos} is a category $\mathcal{E}$ with:
\begin{itemize}
    \item Finite limits
    \item A subobject classifier $\Omega$
    \item Power objects
\end{itemize}
\end{definition}

\subsection{Subobject Classifier}
\begin{definition}
A \textbf{subobject classifier} is an object $\Omega$ with a morphism $\text{true}: 1 \to \Omega$ such that for any monomorphism $m: A \to B$, there exists a unique morphism $\chi_m: B \to \Omega$ making the following diagram a pullback:
$$\begin{tikzcd}
A \arrow[r, "m"] \arrow[d] & B \arrow[d, "\chi_m"] \\
1 \arrow[r, "\text{true}"] & \Omega
\end{tikzcd}$$
\end{definition}

\subsection{Examples of Topoi}
\begin{example}
\begin{itemize}
    \item \textbf{Set}: The category of sets
    \item \textbf{Sh}(X): Sheaves on a topological space $X$
    \item \textbf{Sh}(C, J): Sheaves on a site $(C, J)$
\end{itemize}
\end{example}

\section{Higher Category Theory}

\subsection{2-Categories}
\begin{definition}
A \textbf{2-category} is a category enriched over Cat, consisting of:
\begin{itemize}
    \item Objects
    \item 1-morphisms between objects
    \item 2-morphisms between 1-morphisms
\end{itemize}
with horizontal and vertical composition satisfying interchange laws.
\end{definition}

\subsection{$\infty$-Categories}
\begin{definition}
An \textbf{$\infty$-category} (or $(\infty,1)$-category) is a simplicial set satisfying the weak Kan condition, where morphisms can be composed up to higher homotopies.
\end{definition}

\section{Applications}

\subsection{Algebraic Topology}
\begin{itemize}
    \item Fundamental groupoid
    \item Homology and cohomology as functors
    \item Spectral sequences
    \item Fibrations and cofibrations
\end{itemize}

\subsection{Algebraic Geometry}
\begin{itemize}
    \item Schemes as functors
    \item Sheaves and presheaves
    \item Étale cohomology
    \item Derived categories
\end{itemize}

\subsection{Logic and Computer Science}
\begin{itemize}
    \item Curry-Howard correspondence
    \item Type theory
    \item Domain theory
    \item Coalgebras
\end{itemize}

\subsection{Physics}
\begin{itemize}
    \item Quantum mechanics
    \item String theory
    \item Topological quantum field theory
    \item Categorical quantum mechanics
\end{itemize}

\section{Universal Properties}

\subsection{Initial and Terminal Objects}
\begin{definition}
\begin{itemize}
    \item An object $I$ is \textbf{initial} if for any object $A$, there exists a unique morphism $I \to A$
    \item An object $T$ is \textbf{terminal} if for any object $A$, there exists a unique morphism $A \to T$
\end{itemize}
\end{definition}

\subsection{Universal Elements}
\begin{definition}
A \textbf{universal element} of a functor $F: \mathcal{C}^{\text{op}} \to \text{Set}$ is a pair $(A, x)$ where $A$ is an object and $x \in F(A)$, such that for any other pair $(B, y)$ with $y \in F(B)$, there exists a unique morphism $f: B \to A$ with $F(f)(x) = y$.
\end{definition}

\section{Enriched Categories}

\subsection{Definition}
\begin{definition}
A category \textbf{enriched over} a monoidal category $\mathcal{V}$ is a category $\mathcal{C}$ where:
\begin{itemize}
    \item $\text{Hom}(A,B)$ is an object in $\mathcal{V}$
    \item Composition is a morphism $\text{Hom}(B,C) \otimes \text{Hom}(A,B) \to \text{Hom}(A,C)$
    \item Identity is a morphism $I \to \text{Hom}(A,A)$
\end{itemize}
satisfying associativity and unit laws.
\end{definition}

\subsection{Examples}
\begin{example}
\begin{itemize}
    \item \textbf{Ordinary categories}: Enriched over Set
    \item \textbf{Preorders}: Enriched over $\{0,1\}$
    \item \textbf{Metric spaces}: Enriched over $([0,\infty], \geq)$
    \item \textbf{Abelian categories}: Enriched over Ab
\end{itemize}
\end{example}

\section{Coends and Ends}

\subsection{Definition}
\begin{definition}
For a functor $F: \mathcal{C}^{\text{op}} \times \mathcal{C} \to \mathcal{D}$, the \textbf{coend} is the colimit of the diagram formed by $F(A,A)$ for all objects $A$, with morphisms induced by $F(f, 1_A)$ and $F(1_A, f)$.
\end{definition}

\begin{definition}
For a functor $F: \mathcal{C}^{\text{op}} \times \mathcal{C} \to \mathcal{D}$, the \textbf{end} is the limit of the diagram formed by $F(A,A)$ for all objects $A$, with morphisms induced by $F(f, 1_A)$ and $F(1_A, f)$.
\end{definition}

\subsection{Examples}
\begin{example}
\begin{itemize}
    \item \textbf{Tensor product}: $\int^A F(A) \otimes G(A)$
    \item \textbf{Hom functor}: $\int_A \text{Hom}(F(A), G(A))$
    \item \textbf{Geometric realization}: $\int^n \Delta^n \times X_n$
\end{itemize}
\end{example}

\section{Monoidal Categories}

\subsection{Definition}
\begin{definition}
A \textbf{monoidal category} is a category $\mathcal{C}$ equipped with:
\begin{itemize}
    \item A bifunctor $\otimes: \mathcal{C} \times \mathcal{C} \to \mathcal{C}$
    \item A unit object $I$
    \item Natural isomorphisms for associativity and unit
\end{itemize}
satisfying coherence conditions.
\end{definition}

\subsection{Symmetric Monoidal Categories}
\begin{definition}
A \textbf{symmetric monoidal category} is a monoidal category with a natural isomorphism $\sigma_{A,B}: A \otimes B \to B \otimes A$ satisfying $\sigma_{B,A} \circ \sigma_{A,B} = 1_{A \otimes B}$.
\end{definition}

\subsection{Examples}
\begin{example}
\begin{itemize}
    \item \textbf{Set} with cartesian product
    \item \textbf{Vect} with tensor product
    \item \textbf{Ab} with tensor product
    \item \textbf{Cat} with cartesian product
\end{itemize}
\end{example}

\section{Important Theorems}

\subsection{Adjoint Functor Theorem}
\begin{theorem}[Adjoint Functor Theorem]
A functor $G: \mathcal{D} \to \mathcal{C}$ has a left adjoint if and only if:
\begin{itemize}
    \item $G$ preserves limits
    \item $\mathcal{C}$ is complete
    \item The solution set condition holds
\end{itemize}
\end{theorem}

\subsection{Freyd's Theorem}
\begin{theorem}[Freyd's Theorem]
A small category with all small limits is a preorder.
\end{theorem}

\subsection{Brown Representability}
\begin{theorem}[Brown Representability]
In the homotopy category of pointed CW complexes, a contravariant functor $F$ is representable if and only if:
\begin{itemize}
    \item $F$ satisfies the wedge axiom
    \item $F$ satisfies the Mayer-Vietoris axiom
\end{itemize}
\end{theorem}

\section{Conclusion}

Category theory provides a unifying framework for mathematics by abstracting common patterns across different fields. Key concepts include:

\begin{itemize}
    \item Categories, functors, and natural transformations
    \item Limits, colimits, and universal properties
    \item Adjoint functors and monads
    \item Topoi and higher categories
    \item Enriched categories and monoidal categories
\end{itemize}

These concepts have found applications in:
\begin{itemize}
    \item Algebraic topology and geometry
    \item Logic and computer science
    \item Physics and quantum mechanics
    \item Functional programming
    \item Database theory
\end{itemize}

Category theory continues to be a powerful tool for understanding mathematical structures and their relationships.

\end{document}
