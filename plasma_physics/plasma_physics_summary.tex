\documentclass[11pt]{article}
\usepackage[utf8]{inputenc}
\usepackage{amsmath}
\usepackage{amsfonts}
\usepackage{amssymb}
\usepackage{geometry}
\usepackage{enumitem}
\usepackage{graphicx}
\usepackage{tikz}
\usepackage{pgfplots}
\usepackage{amsthm}
\usepackage{mathtools}

\geometry{margin=1in}

\theoremstyle{definition}
\newtheorem{definition}{Definition}[section]
\newtheorem{theorem}{Theorem}[section]
\newtheorem{lemma}{Lemma}[section]
\newtheorem{corollary}{Corollary}[section]
\newtheorem{example}{Example}[section]
\newtheorem{proposition}{Proposition}[section]

\title{Plasma Physics Summary}
\author{Mathematical Notes}
\date{\today}

\begin{document}

\maketitle

\tableofcontents
\newpage

\section{Introduction to Plasma Physics}

\subsection{What is Plasma?}
\begin{definition}
A \textbf{plasma} is a state of matter consisting of a collection of charged particles (ions and electrons) that is electrically neutral on average and exhibits collective behavior.
\end{definition}

\subsection{Plasma Parameters}
\begin{itemize}
    \item \textbf{Plasma frequency}: $\omega_p = \sqrt{\frac{n_e e^2}{m_e \epsilon_0}}$
    \item \textbf{Debye length}: $\lambda_D = \sqrt{\frac{\epsilon_0 k_B T_e}{n_e e^2}}$
    \item \textbf{Plasma parameter}: $g = \frac{1}{n_e \lambda_D^3}$
    \item \textbf{Collision frequency}: $\nu_{ei} = \frac{n_i e^4 \ln\Lambda}{4\pi \epsilon_0^2 m_e^{1/2} (k_B T_e)^{3/2}}$
\end{itemize}

\subsection{Plasma Conditions}
For a gas to be considered a plasma:
\begin{enumerate}
    \item $\lambda_D \ll L$ (Debye length much smaller than system size)
    \item $N_D = n_e \lambda_D^3 \gg 1$ (many particles in Debye sphere)
    \item $\omega_p \tau \gg 1$ (plasma frequency much larger than collision time)
\end{enumerate}

\section{Magnetohydrodynamics (MHD)}

\subsection{MHD Equations}
The MHD equations describe the macroscopic behavior of plasmas:

\begin{theorem}[MHD Continuity Equation]
$$\frac{\partial \rho}{\partial t} + \nabla \cdot (\rho \vec{v}) = 0$$
\end{theorem}

\begin{theorem}[MHD Momentum Equation]
$$\rho \frac{D\vec{v}}{Dt} = -\nabla p + \vec{J} \times \vec{B} + \rho \vec{g}$$
\end{theorem}

\begin{theorem}[MHD Induction Equation]
$$\frac{\partial \vec{B}}{\partial t} = \nabla \times (\vec{v} \times \vec{B}) + \eta \nabla^2 \vec{B}$$
where $\eta = \frac{1}{\mu_0 \sigma}$ is the magnetic diffusivity.
\end{theorem}

\begin{theorem}[MHD Energy Equation]
$$\frac{D}{Dt}\left(\frac{p}{\rho^{\gamma}}\right) = 0$$
where $\gamma$ is the adiabatic index.
\end{theorem}

\subsection{MHD Approximations}
\begin{itemize}
    \item \textbf{Ideal MHD}: $\sigma \to \infty$ (perfect conductor)
    \item \textbf{Resistive MHD}: Finite conductivity included
    \item \textbf{Hall MHD}: Electron inertia effects included
\end{itemize}

\subsection{Magnetic Reynolds Number}
\begin{definition}
The \textbf{magnetic Reynolds number} is:
$$R_m = \frac{vL}{\eta} = \mu_0 \sigma vL$$
where $v$ is characteristic velocity and $L$ is characteristic length.
\end{definition}

\subsection{Frozen-in Theorem}
\begin{theorem}[Alfvén's Frozen-in Theorem]
In ideal MHD, magnetic field lines are "frozen" into the plasma flow:
$$\frac{d\Phi}{dt} = 0$$
where $\Phi$ is magnetic flux through a surface moving with the plasma.
\end{theorem}

\section{Plasma Waves and Instabilities}

\subsection{Electromagnetic Waves in Plasma}
\begin{theorem}[Dispersion Relation for Electromagnetic Waves]
$$\omega^2 = \omega_p^2 + c^2 k^2$$
\end{theorem}

\subsection{Electrostatic Waves}
\begin{itemize}
    \item \textbf{Electron plasma waves}: $\omega^2 = \omega_p^2 + \frac{3k_B T_e}{m_e}k^2$
    \item \textbf{Ion acoustic waves}: $\omega^2 = \frac{k_B T_e}{m_i}k^2$
    \item \textbf{Upper hybrid waves}: $\omega^2 = \omega_p^2 + \omega_c^2$
    \item \textbf{Lower hybrid waves}: $\omega^2 = \frac{\omega_{ci} \omega_{ce}}{1 + \frac{\omega_p^2}{\omega_c^2}}$
\end{itemize}

\subsection{Magnetohydrodynamic Waves}
\begin{itemize}
    \item \textbf{Alfvén waves}: $v_A = \frac{B}{\sqrt{\mu_0 \rho}}$
    \item \textbf{Magnetosonic waves}: $v_{ms} = \sqrt{v_A^2 + c_s^2}$
    \item \textbf{Slow magnetosonic waves}: $v_s = \frac{v_A c_s}{\sqrt{v_A^2 + c_s^2}}$
\end{itemize}

\subsection{Plasma Instabilities}
\begin{definition}
A \textbf{plasma instability} occurs when small perturbations grow exponentially in time.
\end{definition}

\subsubsection{Rayleigh-Taylor Instability}
\begin{theorem}[Rayleigh-Taylor Growth Rate]
$$\gamma = \sqrt{gk \frac{\rho_2 - \rho_1}{\rho_2 + \rho_1}}$$
where $\rho_2 > \rho_1$ and $g$ is the acceleration.
\end{theorem}

\subsubsection{Kelvin-Helmholtz Instability}
\begin{theorem}[Kelvin-Helmholtz Growth Rate]
$$\gamma = \frac{k \Delta v}{2} \sqrt{\frac{\rho_1 \rho_2}{(\rho_1 + \rho_2)^2}}$$
where $\Delta v$ is the velocity shear.
\end{theorem}

\subsubsection{Two-Stream Instability}
\begin{theorem}[Two-Stream Instability Condition]
$$\omega^2 = \omega_p^2 \left(1 + \frac{v_0^2}{v_{th}^2}\right)$$
where $v_0$ is the relative velocity between streams and $v_{th}$ is thermal velocity.
\end{theorem}

\section{Fusion Physics}

\subsection{Thermonuclear Fusion}
\begin{definition}
\textbf{Thermonuclear fusion} is the process of combining light atomic nuclei to form heavier nuclei, releasing energy.
\end{definition}

\subsection{Fusion Reactions}
\begin{itemize}
    \item \textbf{Deuterium-Tritium}: $D + T \to ^4He + n + 17.6$ MeV
    \item \textbf{Deuterium-Deuterium}: $D + D \to ^3He + n + 3.3$ MeV
    \item \textbf{Deuterium-Deuterium}: $D + D \to T + p + 4.0$ MeV
    \item \textbf{Proton-Proton}: $p + p \to D + e^+ + \nu_e + 0.42$ MeV
\end{itemize}

\subsection{Fusion Conditions}
\begin{theorem}[Lawson Criterion]
For energy breakeven:
$$n\tau \geq \frac{12k_B T}{\langle \sigma v \rangle E_f}$$
where $n$ is density, $\tau$ is confinement time, $T$ is temperature, $\langle \sigma v \rangle$ is reaction rate, and $E_f$ is fusion energy per reaction.
\end{theorem}

\subsection{Tokamak Physics}
\begin{definition}
A \textbf{tokamak} is a toroidal magnetic confinement device for controlled fusion.
\end{definition}

\subsubsection{Magnetic Confinement}
\begin{itemize}
    \item \textbf{Toroidal field}: $B_\phi = \frac{B_0 R_0}{R}$
    \item \textbf{Poloidal field}: Generated by plasma current
    \item \textbf{Safety factor}: $q = \frac{r B_\phi}{R B_\theta}$
\end{itemize}

\subsubsection{Plasma Equilibrium}
\begin{theorem}[Grad-Shafranov Equation]
$$\Delta^* \psi = -\mu_0 R^2 \frac{dp}{d\psi} - F \frac{dF}{d\psi}$$
where $\psi$ is the poloidal flux function and $F = RB_\phi$.
\end{theorem}

\subsection{Fusion Power Density}
\begin{theorem}[Fusion Power Density]
$$P_f = \frac{1}{4}n^2 \langle \sigma v \rangle E_f$$
\end{theorem}

\section{Space and Astrophysical Plasmas}

\subsection{Solar Wind}
\begin{definition}
The \textbf{solar wind} is a stream of charged particles ejected from the Sun's corona.
\end{definition}

\subsubsection{Solar Wind Properties}
\begin{itemize}
    \item \textbf{Slow solar wind}: $v \approx 400$ km/s, $n \approx 10^7$ cm$^{-3}$
    \item \textbf{Fast solar wind}: $v \approx 800$ km/s, $n \approx 3 \times 10^6$ cm$^{-3}$
    \item \textbf{Temperature}: $T \approx 10^5 - 10^6$ K
\end{itemize}

\subsection{Magnetosphere}
\begin{definition}
The \textbf{magnetosphere} is the region around a planet dominated by its magnetic field.
\end{definition}

\subsubsection{Earth's Magnetosphere}
\begin{itemize}
    \item \textbf{Bow shock}: Standoff distance $\approx 10-15 R_E$
    \item \textbf{Magnetopause}: Boundary between solar wind and magnetosphere
    \item \textbf{Magnetotail}: Extended region downstream from Earth
\end{itemize}

\subsection{Stellar Atmospheres}
\begin{itemize}
    \item \textbf{Photosphere}: Visible surface, $T \approx 5800$ K
    \item \textbf{Chromosphere}: Transition region, $T \approx 10^4$ K
    \item \textbf{Corona}: Outer atmosphere, $T \approx 10^6$ K
\end{itemize}

\subsection{Interstellar Medium}
\begin{itemize}
    \item \textbf{Cold neutral medium}: $T \approx 100$ K, $n \approx 30$ cm$^{-3}$
    \item \textbf{Warm neutral medium}: $T \approx 8000$ K, $n \approx 0.3$ cm$^{-3}$
    \item \textbf{Warm ionized medium}: $T \approx 8000$ K, $n \approx 0.1$ cm$^{-3}$
    \item \textbf{Hot ionized medium}: $T \approx 10^6$ K, $n \approx 0.003$ cm$^{-3}$
\end{itemize}

\section{Computational Plasma Physics}

\subsection{Particle-in-Cell (PIC) Method}
\begin{definition}
The \textbf{Particle-in-Cell method} simulates plasmas by following individual particles in self-consistent electromagnetic fields.
\end{definition}

\subsubsection{PIC Algorithm}
\begin{enumerate}
    \item Initialize particles and fields
    \item Push particles using Lorentz force
    \item Deposit charge and current densities
    \item Solve Maxwell's equations for fields
    \item Repeat for next time step
\end{enumerate}

\subsection{Magnetohydrodynamic Simulations}
\begin{itemize}
    \item \textbf{Finite difference methods}: Direct discretization of MHD equations
    \item \textbf{Finite volume methods}: Conservative formulation
    \item \textbf{Spectral methods}: Fourier decomposition
    \item \textbf{Adaptive mesh refinement}: Variable resolution
\end{itemize}

\subsection{Kinetic Simulations}
\begin{itemize}
    \item \textbf{Vlasov equation}: $\frac{\partial f}{\partial t} + \vec{v} \cdot \nabla f + \frac{q}{m}(\vec{E} + \vec{v} \times \vec{B}) \cdot \nabla_v f = 0$
    \item \textbf{Gyrokinetic theory}: Averaged over gyromotion
    \item \textbf{Drift kinetic theory}: Includes drift motion
\end{itemize}

\section{Plasma Diagnostics}

\subsection{Electromagnetic Diagnostics}
\begin{itemize}
    \item \textbf{Magnetic probes}: Measure magnetic field fluctuations
    \item \textbf{Electric probes}: Measure electric field and plasma potential
    \item \textbf{Interferometry}: Measure electron density
    \item \textbf{Thomson scattering}: Measure electron temperature and density
\end{itemize}

\subsection{Particle Diagnostics}
\begin{itemize}
    \item \textbf{Langmuir probes}: Current-voltage characteristics
    \item \textbf{Retarding field analyzers}: Energy distribution functions
    \item \textbf{Neutral particle analyzers}: Ion temperature and density
    \item \textbf{Fast ion diagnostics}: Energetic particle measurements
\end{itemize}

\subsection{Spectroscopic Diagnostics}
\begin{itemize}
    \item \textbf{Optical emission spectroscopy}: Line intensities and widths
    \item \textbf{X-ray spectroscopy}: Bremsstrahlung and line emission
    \item \textbf{Neutron diagnostics}: Fusion product measurements
\end{itemize}

\section{Applications}

\subsection{Fusion Energy}
\begin{itemize}
    \item \textbf{Magnetic confinement fusion}: Tokamaks, stellarators
    \item \textbf{Inertial confinement fusion}: Laser and particle beam drivers
    \item \textbf{Magnetized target fusion}: Hybrid approaches
\end{itemize}

\subsection{Space Physics}
\begin{itemize}
    \item \textbf{Solar-terrestrial interactions}: Space weather
    \item \textbf{Planetary magnetospheres}: Comparative planetology
    \item \textbf{Astrophysical plasmas}: Stellar and galactic physics
\end{itemize}

\subsection{Industrial Applications}
\begin{itemize}
    \item \textbf{Plasma processing}: Semiconductor manufacturing
    \item \textbf{Plasma propulsion}: Electric spacecraft propulsion
    \item \textbf{Plasma medicine}: Medical applications
    \item \textbf{Materials processing}: Surface modification
\end{itemize}

\section{Important Constants and Parameters}

\subsection{Fundamental Constants}
\begin{itemize}
    \item Electron charge: $e = 1.602 \times 10^{-19}$ C
    \item Electron mass: $m_e = 9.109 \times 10^{-31}$ kg
    \item Proton mass: $m_p = 1.673 \times 10^{-27}$ kg
    \item Permittivity of free space: $\epsilon_0 = 8.854 \times 10^{-12}$ F/m
    \item Permeability of free space: $\mu_0 = 4\pi \times 10^{-7}$ H/m
    \item Boltzmann constant: $k_B = 1.381 \times 10^{-23}$ J/K
    \item Speed of light: $c = 2.998 \times 10^8$ m/s
\end{itemize}

\subsection{Plasma Parameters}
\begin{itemize}
    \item Classical electron radius: $r_e = \frac{e^2}{4\pi \epsilon_0 m_e c^2} = 2.818 \times 10^{-15}$ m
    \item Bohr radius: $a_0 = \frac{4\pi \epsilon_0 \hbar^2}{m_e e^2} = 5.292 \times 10^{-11}$ m
    \item Fine structure constant: $\alpha = \frac{e^2}{4\pi \epsilon_0 \hbar c} = 7.297 \times 10^{-3}$
    \item Electron cyclotron frequency: $\omega_{ce} = \frac{eB}{m_e}$
    \item Ion cyclotron frequency: $\omega_{ci} = \frac{eB}{m_i}$
\end{itemize}

\subsection{Fusion Parameters}
\begin{itemize}
    \item DT fusion cross-section peak: $\sigma_{max} \approx 5 \times 10^{-28}$ m$^2$ at $T \approx 100$ keV
    \item Lawson criterion for DT: $n\tau \geq 10^{20}$ s/m$^3$ at $T \approx 10$ keV
    \item ITER parameters: $R = 6.2$ m, $a = 2.0$ m, $B = 5.3$ T, $I_p = 15$ MA
\end{itemize}

\end{document}
