\documentclass[11pt]{article}
\usepackage[utf8]{inputenc}
\usepackage{amsmath}
\usepackage{amsfonts}
\usepackage{amssymb}
\usepackage{geometry}
\usepackage{enumitem}
\usepackage{graphicx}
\usepackage{tikz}
\usepackage{pgfplots}
\usepackage{amsthm}
\usepackage{mathtools}

\geometry{margin=1in}

\theoremstyle{definition}
\newtheorem{definition}{Definition}[section]
\newtheorem{theorem}{Theorem}[section]
\newtheorem{lemma}{Lemma}[section]
\newtheorem{corollary}{Corollary}[section]
\newtheorem{example}{Example}[section]
\newtheorem{proposition}{Proposition}[section]

\title{Signal Processing Summary}
\author{Mathematical Notes}
\date{\today}

\begin{document}

\maketitle

\tableofcontents
\newpage

\section{Discrete-Time Signals}

\subsection{Basic Definitions}
\begin{definition}
A \textbf{discrete-time signal} is a sequence $x[n]$ where $n \in \mathbb{Z}$ is the discrete time index.
\end{definition}

\begin{definition}
A signal is \textbf{causal} if $x[n] = 0$ for $n < 0$.
\end{definition}

\begin{definition}
A signal is \textbf{finite-length} if $x[n] = 0$ for $n < 0$ and $n \geq N$.
\end{definition}

\subsection{Elementary Signals}
\begin{itemize}
    \item \textbf{Unit impulse}: $\delta[n] = \begin{cases} 1 & n = 0 \\ 0 & n \neq 0 \end{cases}$
    \item \textbf{Unit step}: $u[n] = \begin{cases} 1 & n \geq 0 \\ 0 & n < 0 \end{cases}$
    \item \textbf{Complex exponential}: $e^{j\omega_0 n}$
    \item \textbf{Sinusoid}: $\cos(\omega_0 n + \phi)$
\end{itemize}

\subsection{Signal Properties}
\begin{definition}
A signal is \textbf{periodic} with period $N$ if $x[n] = x[n + N]$ for all $n$.
\end{definition}

\begin{definition}
The \textbf{energy} of a signal is $E_x = \sum_{n=-\infty}^{\infty} |x[n]|^2$.
\end{definition}

\begin{definition}
The \textbf{power} of a periodic signal with period $N$ is $P_x = \frac{1}{N} \sum_{n=0}^{N-1} |x[n]|^2$.
\end{definition}

\section{Linear Time-Invariant Systems}

\subsection{System Properties}
\begin{definition}
A system is \textbf{linear} if $T[ax_1[n] + bx_2[n]] = aT[x_1[n]] + bT[x_2[n]]$.
\end{definition}

\begin{definition}
A system is \textbf{time-invariant} if $T[x[n-k]] = y[n-k]$ for any $k$.
\end{definition}

\subsection{Impulse Response}
\begin{definition}
The \textbf{impulse response} of an LTI system is $h[n] = T[\delta[n]]$.
\end{definition}

\begin{theorem}[Convolution Sum]
For an LTI system with impulse response $h[n]$, the output is:
$$y[n] = x[n] * h[n] = \sum_{k=-\infty}^{\infty} x[k]h[n-k]$$
\end{theorem}

\subsection{System Properties from Impulse Response}
\begin{itemize}
    \item \textbf{Causal}: $h[n] = 0$ for $n < 0$
    \item \textbf{Stable}: $\sum_{n=-\infty}^{\infty} |h[n]| < \infty$
    \item \textbf{FIR}: Finite impulse response (finite length)
    \item \textbf{IIR}: Infinite impulse response (infinite length)
\end{itemize}

\section{Discrete Fourier Transform}

\subsection{Definition}
\begin{definition}
The \textbf{Discrete Fourier Transform} (DFT) of a length-$N$ sequence $x[n]$ is:
$$X[k] = \sum_{n=0}^{N-1} x[n] e^{-j2\pi kn/N}, \quad k = 0, 1, \ldots, N-1$$
\end{definition}

\begin{definition}
The \textbf{Inverse DFT} is:
$$x[n] = \frac{1}{N} \sum_{k=0}^{N-1} X[k] e^{j2\pi kn/N}, \quad n = 0, 1, \ldots, N-1$$
\end{definition}

\subsection{DFT Properties}
\begin{itemize}
    \item \textbf{Linearity}: $\text{DFT}[ax[n] + by[n]] = aX[k] + bY[k]$
    \item \textbf{Circular shift}: $\text{DFT}[x[(n-m)_N]] = X[k]e^{-j2\pi km/N}$
    \item \textbf{Parseval's theorem}: $\sum_{n=0}^{N-1} |x[n]|^2 = \frac{1}{N} \sum_{k=0}^{N-1} |X[k]|^2$
    \item \textbf{Circular convolution}: $\text{DFT}[x[n] \circledast y[n]] = X[k]Y[k]$
\end{itemize}

\subsection{Fast Fourier Transform}
\begin{theorem}[FFT Algorithm]
The DFT can be computed in $O(N \log N)$ operations using the FFT algorithm.
\end{theorem}

\section{Z-Transform}

\subsection{Definition}
\begin{definition}
The \textbf{Z-transform} of a sequence $x[n]$ is:
$$X(z) = \sum_{n=-\infty}^{\infty} x[n] z^{-n}$$
\end{definition}

\subsection{Region of Convergence}
\begin{definition}
The \textbf{Region of Convergence} (ROC) is the set of $z$ values for which the Z-transform converges.
\end{definition}

\subsection{Z-Transform Properties}
\begin{itemize}
    \item \textbf{Linearity}: $\mathcal{Z}[ax[n] + by[n]] = aX(z) + bY(z)$
    \item \textbf{Time shift}: $\mathcal{Z}[x[n-k]] = z^{-k}X(z)$
    \item \textbf{Convolution}: $\mathcal{Z}[x[n] * y[n]] = X(z)Y(z)$
    \item \textbf{Multiplication by $n$}: $\mathcal{Z}[nx[n]] = -z \frac{dX(z)}{dz}$
\end{itemize}

\subsection{Inverse Z-Transform}
\begin{itemize}
    \item \textbf{Partial fraction expansion}
    \item \textbf{Power series expansion}
    \item \textbf{Contour integration}: $x[n] = \frac{1}{2\pi j} \oint_C X(z) z^{n-1} dz$
\end{itemize}

\section{Digital Filters}

\subsection{FIR Filters}
\begin{definition}
A \textbf{Finite Impulse Response} filter has impulse response:
$$h[n] = \begin{cases} b_n & 0 \leq n \leq M \\ 0 & \text{otherwise} \end{cases}$$
\end{definition}

The system function is:
$$H(z) = \sum_{n=0}^{M} b_n z^{-n}$$

\subsection{IIR Filters}
\begin{definition}
An \textbf{Infinite Impulse Response} filter has system function:
$$H(z) = \frac{\sum_{k=0}^{M} b_k z^{-k}}{1 + \sum_{k=1}^{N} a_k z^{-k}}$$
\end{definition}

The difference equation is:
$$y[n] = \sum_{k=0}^{M} b_k x[n-k] - \sum_{k=1}^{N} a_k y[n-k]$$

\subsection{Filter Design}
\begin{itemize}
    \item \textbf{Window method} for FIR filters
    \item \textbf{Impulse invariance} for IIR filters
    \item \textbf{Bilinear transform} for IIR filters
    \item \textbf{Least squares} design
\end{itemize}

\section{Frequency Response}

\subsection{Definition}
\begin{definition}
The \textbf{frequency response} of an LTI system is:
$$H(e^{j\omega}) = \sum_{n=-\infty}^{\infty} h[n] e^{-j\omega n}$$
\end{definition}

\subsection{Magnitude and Phase}
$$H(e^{j\omega}) = |H(e^{j\omega})| e^{j\angle H(e^{j\omega})}$$

\subsection{Filter Types}
\begin{itemize}
    \item \textbf{Lowpass}: Passes low frequencies, attenuates high frequencies
    \item \textbf{Highpass}: Passes high frequencies, attenuates low frequencies
    \item \textbf{Bandpass}: Passes frequencies in a specific band
    \item \textbf{Bandstop}: Attenuates frequencies in a specific band
\end{itemize}

\section{Sampling Theory}

\subsection{Nyquist-Shannon Sampling Theorem}
\begin{theorem}[Sampling Theorem]
A bandlimited signal with maximum frequency $f_m$ can be perfectly reconstructed from its samples if the sampling rate $f_s \geq 2f_m$.
\end{theorem}

\subsection{Aliasing}
\begin{definition}
\textbf{Aliasing} occurs when the sampling rate is insufficient, causing high-frequency components to appear as low-frequency components.
\end{definition}

\subsection{Anti-aliasing Filter}
An anti-aliasing filter is a lowpass filter applied before sampling to prevent aliasing.

\section{Multirate Signal Processing}

\subsection{Downsampling}
\begin{definition}
\textbf{Downsampling} by factor $M$: $y[n] = x[Mn]$
\end{definition}

In frequency domain:
$$Y(e^{j\omega}) = \frac{1}{M} \sum_{k=0}^{M-1} X(e^{j(\omega-2\pi k)/M})$$

\subsection{Upsampling}
\begin{definition}
\textbf{Upsampling} by factor $L$: $y[n] = \begin{cases} x[n/L] & n = 0, \pm L, \pm 2L, \ldots \\ 0 & \text{otherwise} \end{cases}$
\end{definition}

In frequency domain:
$$Y(e^{j\omega}) = X(e^{jL\omega})$$

\subsection{Rational Sampling Rate Conversion}
To change sampling rate by factor $L/M$:
\begin{enumerate}
    \item Upsample by $L$
    \item Filter to prevent aliasing
    \item Downsample by $M$
\end{enumerate}

\section{Adaptive Filters}

\subsection{Least Mean Squares (LMS)}
\begin{definition}
The \textbf{LMS algorithm} updates filter coefficients as:
$$\mathbf{w}[n+1] = \mathbf{w}[n] + \mu e[n] \mathbf{x}[n]$$
where $\mu$ is the step size and $e[n]$ is the error signal.
\end{definition}

\subsection{Recursive Least Squares (RLS)}
\begin{definition}
The \textbf{RLS algorithm} minimizes the weighted least squares cost function:
$$J[n] = \sum_{i=0}^{n} \lambda^{n-i} |d[i] - \mathbf{w}^T[n] \mathbf{x}[i]|^2$$
where $\lambda$ is the forgetting factor.
\end{definition}

\section{Statistical Signal Processing}

\subsection{Power Spectral Density}
\begin{definition}
The \textbf{power spectral density} of a wide-sense stationary process is:
$$S_{xx}(e^{j\omega}) = \sum_{m=-\infty}^{\infty} R_{xx}[m] e^{-j\omega m}$$
where $R_{xx}[m] = E[x[n]x^*[n-m]]$ is the autocorrelation function.
\end{definition}

\subsection{Wiener Filter}
\begin{definition}
The \textbf{Wiener filter} minimizes the mean-square error between the desired signal and filter output.
\end{definition}

For FIR Wiener filter:
$$\mathbf{w}_{\text{opt}} = \mathbf{R}_{xx}^{-1} \mathbf{r}_{xd}$$
where $\mathbf{R}_{xx}$ is the autocorrelation matrix and $\mathbf{r}_{xd}$ is the cross-correlation vector.

\section{Image Processing}

\subsection{2D Discrete Fourier Transform}
\begin{definition}
The \textbf{2D DFT} of an $M \times N$ image $f[m,n]$ is:
$$F[u,v] = \sum_{m=0}^{M-1} \sum_{n=0}^{N-1} f[m,n] e^{-j2\pi(um/M + vn/N)}$$
\end{definition}

\subsection{Image Filtering}
\begin{itemize}
    \item \textbf{Spatial domain}: Convolution with filter kernel
    \item \textbf{Frequency domain}: Multiplication with frequency response
    \item \textbf{Edge detection}: Sobel, Prewitt, Laplacian operators
    \item \textbf{Smoothing}: Gaussian, median filtering
\end{itemize}

\section{Compression}

\subsection{Lossless Compression}
\begin{itemize}
    \item \textbf{Huffman coding}: Variable-length coding based on symbol probabilities
    \item \textbf{Lempel-Ziv}: Dictionary-based compression
    \item \textbf{Predictive coding}: Encode prediction errors
\end{itemize}

\subsection{Lossy Compression}
\begin{itemize}
    \item \textbf{Transform coding}: DCT, wavelet transforms
    \item \textbf{Quantization}: Reduce precision of coefficients
    \item \textbf{JPEG}: DCT-based image compression
    \item \textbf{MPEG}: Motion-compensated video compression
\end{itemize}

\section{Applications}

\subsection{Communication Systems}
\begin{itemize}
    \item Modulation and demodulation
    \item Channel equalization
    \item Error correction coding
    \item Synchronization
\end{itemize}

\subsection{Audio Processing}
\begin{itemize}
    \item Speech recognition
    \item Audio compression (MP3, AAC)
    \item Noise reduction
    \item Echo cancellation
\end{itemize}

\subsection{Biomedical Signal Processing}
\begin{itemize}
    \item ECG analysis
    \item EEG signal processing
    \item Medical image analysis
    \item Heart rate variability
\end{itemize}

\subsection{Radar and Sonar}
\begin{itemize}
    \item Target detection
    \item Range and velocity estimation
    \item Beamforming
    \item Clutter suppression
\end{itemize}

\section{Important Algorithms}

\subsection{Fast Convolution}
\begin{itemize}
    \item \textbf{Overlap-add method}
    \item \textbf{Overlap-save method}
    \item Use FFT for efficient computation
\end{itemize}

\subsection{Filter Banks}
\begin{itemize}
    \item \textbf{Analysis filter bank}: Decompose signal into subbands
    \item \textbf{Synthesis filter bank}: Reconstruct signal from subbands
    \item \textbf{Perfect reconstruction}: Input equals output
\end{itemize}

\subsection{Wavelets}
\begin{itemize}
    \item \textbf{Continuous wavelet transform}
    \item \textbf{Discrete wavelet transform}
    \item \textbf{Wavelet packet decomposition}
    \item Applications in compression and denoising
\end{itemize}

\section{Implementation Considerations}

\subsection{Finite Wordlength Effects}
\begin{itemize}
    \item \textbf{Quantization noise}
    \item \textbf{Overflow and underflow}
    \item \textbf{Limit cycles}
    \item \textbf{Coefficient sensitivity}
\end{itemize}

\subsection{Computational Complexity}
\begin{itemize}
    \item \textbf{FIR filters}: $O(N)$ per output sample
    \item \textbf{IIR filters}: $O(N)$ per output sample
    \item \textbf{FFT}: $O(N \log N)$ for length-$N$ transform
    \item \textbf{Convolution}: $O(NM)$ for length-$N$ and length-$M$ sequences
\end{itemize}

\section{Key Theorems}

\subsection{Parseval's Theorem}
\begin{theorem}
$$\sum_{n=-\infty}^{\infty} |x[n]|^2 = \frac{1}{2\pi} \int_{-\pi}^{\pi} |X(e^{j\omega})|^2 d\omega$$
\end{theorem}

\subsection{Convolution Theorem}
\begin{theorem}
$$\mathcal{F}[x[n] * y[n]] = X(e^{j\omega}) Y(e^{j\omega})$$
\end{theorem}

\subsection{Modulation Theorem}
\begin{theorem}
$$\mathcal{F}[x[n] e^{j\omega_0 n}] = X(e^{j(\omega - \omega_0)})$$
\end{theorem}

\section{Important Transforms}

\subsection{Discrete Cosine Transform}
\begin{definition}
The \textbf{DCT} is defined as:
$$X[k] = \sum_{n=0}^{N-1} x[n] \cos\left(\frac{\pi(2n+1)k}{2N}\right)$$
\end{definition}

\subsection{Walsh-Hadamard Transform}
\begin{definition}
The \textbf{WHT} uses Walsh functions as basis functions and is computationally efficient.
\end{definition}

\subsection{Karhunen-Loève Transform}
\begin{definition}
The \textbf{KLT} is the optimal transform for decorrelating signals with known statistics.
\end{definition}

\end{document}
