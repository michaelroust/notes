\documentclass[11pt]{article}
\usepackage[utf8]{inputenc}
\usepackage{amsmath}
\usepackage{amsfonts}
\usepackage{amssymb}
\usepackage{geometry}
\usepackage{enumitem}
\usepackage{graphicx}
\usepackage{tikz}
\usepackage{pgfplots}
\usepackage{amsthm}
\usepackage{mathtools}

\geometry{margin=1in}

\theoremstyle{definition}
\newtheorem{definition}{Definition}[section]
\newtheorem{theorem}{Theorem}[section]
\newtheorem{lemma}{Lemma}[section]
\newtheorem{corollary}{Corollary}[section]
\newtheorem{example}{Example}[section]

\title{Advanced Calculus Summary}
\author{Mathematical Notes}
\date{\today}

\begin{document}

\maketitle

\tableofcontents
\newpage

\section{Multivariable Functions}

\subsection{Limits and Continuity}
\begin{definition}
A function $f: \mathbb{R}^n \to \mathbb{R}$ has limit $L$ at point $\mathbf{a}$ if for every $\epsilon > 0$, there exists $\delta > 0$ such that:
$$0 < \|\mathbf{x} - \mathbf{a}\| < \delta \Rightarrow |f(\mathbf{x}) - L| < \epsilon$$
\end{definition}

\subsection{Partial Derivatives}
\begin{definition}
The \textbf{partial derivative} of $f(x,y)$ with respect to $x$ is:
$$\frac{\partial f}{\partial x} = \lim_{h \to 0} \frac{f(x+h,y) - f(x,y)}{h}$$
\end{definition}

\subsection{Clairaut's Theorem}
\begin{theorem}
If $f$ has continuous second partial derivatives, then:
$$\frac{\partial^2 f}{\partial x \partial y} = \frac{\partial^2 f}{\partial y \partial x}$$
\end{theorem}

\subsection{Chain Rule for Multivariable Functions}
\begin{theorem}
If $z = f(x,y)$ where $x = g(t)$ and $y = h(t)$, then:
$$\frac{dz}{dt} = \frac{\partial f}{\partial x} \frac{dx}{dt} + \frac{\partial f}{\partial y} \frac{dy}{dt}$$
\end{theorem}

\subsection{Implicit Differentiation}
For $F(x,y,z) = 0$:
$$\frac{\partial z}{\partial x} = -\frac{F_x}{F_z}, \quad \frac{\partial z}{\partial y} = -\frac{F_y}{F_z}$$

\section{Directional Derivatives and Gradients}

\subsection{Directional Derivative}
\begin{definition}
The \textbf{directional derivative} of $f$ at $\mathbf{a}$ in direction $\mathbf{u}$ is:
$$D_{\mathbf{u}} f(\mathbf{a}) = \lim_{h \to 0} \frac{f(\mathbf{a} + h\mathbf{u}) - f(\mathbf{a})}{h}$$
\end{definition}

\subsection{Gradient}
\begin{definition}
The \textbf{gradient} of $f(x,y,z)$ is:
$$\nabla f = \left(\frac{\partial f}{\partial x}, \frac{\partial f}{\partial y}, \frac{\partial f}{\partial z}\right)$$
\end{definition}

\subsection{Relationship}
\begin{theorem}
$$D_{\mathbf{u}} f = \nabla f \cdot \mathbf{u}$$
where $\mathbf{u}$ is a unit vector.
\end{theorem}

\subsection{Properties of Gradient}
\begin{itemize}
    \item $\nabla f$ points in the direction of maximum increase
    \item $|\nabla f|$ is the maximum rate of change
    \item $\nabla f$ is perpendicular to level curves/surfaces
\end{itemize}

\section{Tangent Planes and Linear Approximation}

\subsection{Tangent Plane}
For surface $z = f(x,y)$ at point $(a,b,f(a,b))$:
$$z = f(a,b) + f_x(a,b)(x-a) + f_y(a,b)(y-b)$$

\subsection{Linear Approximation}
$$f(x,y) \approx f(a,b) + f_x(a,b)(x-a) + f_y(a,b)(y-b)$$

\subsection{Total Differential}
$$dz = \frac{\partial f}{\partial x} dx + \frac{\partial f}{\partial y} dy$$

\section{Maximum and Minimum Values}

\subsection{Critical Points}
\begin{definition}
A \textbf{critical point} of $f(x,y)$ is a point $(a,b)$ where either:
\begin{itemize}
    \item $f_x(a,b) = 0$ and $f_y(a,b) = 0$, or
    \item $f_x(a,b)$ or $f_y(a,b)$ doesn't exist
\end{itemize}
\end{definition}

\subsection{Second Derivative Test}
\begin{theorem}
Let $D = f_{xx}(a,b)f_{yy}(a,b) - [f_{xy}(a,b)]^2$ at critical point $(a,b)$:
\begin{itemize}
    \item If $D > 0$ and $f_{xx}(a,b) > 0$, then $(a,b)$ is a local minimum
    \item If $D > 0$ and $f_{xx}(a,b) < 0$, then $(a,b)$ is a local maximum
    \item If $D < 0$, then $(a,b)$ is a saddle point
    \item If $D = 0$, the test is inconclusive
\end{itemize}
\end{theorem}

\subsection{Lagrange Multipliers}
\begin{theorem}
To find extrema of $f(x,y,z)$ subject to constraint $g(x,y,z) = k$:
$$\nabla f = \lambda \nabla g$$
for some scalar $\lambda$.
\end{theorem}

\section{Multiple Integrals}

\subsection{Double Integrals}
\begin{definition}
$$\iint_R f(x,y) \, dA = \int_a^b \int_{g_1(x)}^{g_2(x)} f(x,y) \, dy \, dx$$
\end{definition}

\subsection{Properties of Double Integrals}
\begin{itemize}
    \item $\iint_R [f(x,y) + g(x,y)] \, dA = \iint_R f(x,y) \, dA + \iint_R g(x,y) \, dA$
    \item $\iint_R cf(x,y) \, dA = c\iint_R f(x,y) \, dA$
    \item If $f(x,y) \geq g(x,y)$ on $R$, then $\iint_R f(x,y) \, dA \geq \iint_R g(x,y) \, dA$
\end{itemize}

\subsection{Polar Coordinates}
$$\iint_R f(x,y) \, dA = \int_\alpha^\beta \int_{h_1(\theta)}^{h_2(\theta)} f(r\cos\theta, r\sin\theta) \, r \, dr \, d\theta$$

\subsection{Triple Integrals}
$$\iiint_E f(x,y,z) \, dV = \int_a^b \int_{g_1(x)}^{g_2(x)} \int_{u_1(x,y)}^{u_2(x,y)} f(x,y,z) \, dz \, dy \, dx$$

\subsection{Cylindrical Coordinates}
$$x = r\cos\theta, \quad y = r\sin\theta, \quad z = z$$
$$dV = r \, dz \, dr \, d\theta$$

\subsection{Spherical Coordinates}
$$x = \rho\sin\phi\cos\theta, \quad y = \rho\sin\phi\sin\theta, \quad z = \rho\cos\phi$$
$$dV = \rho^2\sin\phi \, d\rho \, d\phi \, d\theta$$

\section{Vector Fields}

\subsection{Definition}
\begin{definition}
A \textbf{vector field} on $\mathbb{R}^n$ is a function $\mathbf{F}$ that assigns to each point $\mathbf{x}$ a vector $\mathbf{F}(\mathbf{x})$.
\end{definition}

\subsection{Gradient Fields}
\begin{definition}
A vector field $\mathbf{F}$ is \textbf{conservative} if $\mathbf{F} = \nabla f$ for some scalar function $f$.
\end{definition}

\subsection{Divergence}
\begin{definition}
$$\text{div } \mathbf{F} = \nabla \cdot \mathbf{F} = \frac{\partial P}{\partial x} + \frac{\partial Q}{\partial y} + \frac{\partial R}{\partial z}$$
\end{definition}

\subsection{Curl}
\begin{definition}
$$\text{curl } \mathbf{F} = \nabla \times \mathbf{F} = \begin{vmatrix}
\mathbf{i} & \mathbf{j} & \mathbf{k} \\
\frac{\partial}{\partial x} & \frac{\partial}{\partial y} & \frac{\partial}{\partial z} \\
P & Q & R
\end{vmatrix}$$
\end{definition}

\section{Line Integrals}

\subsection{Line Integral of Scalar Function}
$$\int_C f(x,y,z) \, ds = \int_a^b f(\mathbf{r}(t)) |\mathbf{r}'(t)| \, dt$$

\subsection{Line Integral of Vector Field}
$$\int_C \mathbf{F} \cdot d\mathbf{r} = \int_a^b \mathbf{F}(\mathbf{r}(t)) \cdot \mathbf{r}'(t) \, dt$$

\subsection{Fundamental Theorem for Line Integrals}
\begin{theorem}
If $\mathbf{F} = \nabla f$ and $C$ is any curve from $A$ to $B$, then:
$$\int_C \mathbf{F} \cdot d\mathbf{r} = f(B) - f(A)$$
\end{theorem}

\subsection{Independence of Path}
\begin{theorem}
$\int_C \mathbf{F} \cdot d\mathbf{r}$ is independent of path if and only if $\mathbf{F}$ is conservative.
\end{theorem}

\section{Green's Theorem}

\subsection{Green's Theorem}
\begin{theorem}
$$\oint_C P \, dx + Q \, dy = \iint_D \left(\frac{\partial Q}{\partial x} - \frac{\partial P}{\partial y}\right) \, dA$$
where $C$ is the positively oriented boundary of region $D$.
\end{theorem}

\subsection{Area Using Green's Theorem}
$$A = \frac{1}{2} \oint_C x \, dy - y \, dx$$

\section{Surface Integrals}

\subsection{Parametric Surfaces}
A surface $S$ can be parametrized as:
$$\mathbf{r}(u,v) = x(u,v)\mathbf{i} + y(u,v)\mathbf{j} + z(u,v)\mathbf{k}$$

\subsection{Surface Area}
$$A = \iint_D |\mathbf{r}_u \times \mathbf{r}_v| \, dA$$

\subsection{Surface Integral of Scalar Function}
$$\iint_S f(x,y,z) \, dS = \iint_D f(\mathbf{r}(u,v)) |\mathbf{r}_u \times \mathbf{r}_v| \, dA$$

\subsection{Surface Integral of Vector Field}
$$\iint_S \mathbf{F} \cdot d\mathbf{S} = \iint_S \mathbf{F} \cdot \mathbf{n} \, dS$$

\section{Divergence Theorem}

\subsection{Divergence Theorem (Gauss's Theorem)}
\begin{theorem}
$$\iint_S \mathbf{F} \cdot d\mathbf{S} = \iiint_E \nabla \cdot \mathbf{F} \, dV$$
where $S$ is the boundary of solid region $E$.
\end{theorem}

\section{Stokes' Theorem}

\subsection{Stokes' Theorem}
\begin{theorem}
$$\oint_C \mathbf{F} \cdot d\mathbf{r} = \iint_S (\nabla \times \mathbf{F}) \cdot d\mathbf{S}$$
where $C$ is the boundary of surface $S$.
\end{theorem}

\section{Sequences and Series of Functions}

\subsection{Pointwise Convergence}
\begin{definition}
Sequence $\{f_n\}$ converges pointwise to $f$ if:
$$\lim_{n \to \infty} f_n(x) = f(x) \text{ for each } x \in D$$
\end{definition}

\subsection{Uniform Convergence}
\begin{definition}
Sequence $\{f_n\}$ converges uniformly to $f$ if:
$$\lim_{n \to \infty} \sup_{x \in D} |f_n(x) - f(x)| = 0$$
\end{definition}

\subsection{Weierstrass M-Test}
\begin{theorem}
If $|f_n(x)| \leq M_n$ for all $x \in D$ and $\sum M_n$ converges, then $\sum f_n(x)$ converges uniformly.
\end{theorem}

\section{Power Series}

\subsection{Radius of Convergence}
\begin{theorem}
For power series $\sum_{n=0}^{\infty} a_n(x-c)^n$:
$$R = \frac{1}{\limsup_{n \to \infty} |a_n|^{1/n}}$$
\end{theorem}

\subsection{Operations on Power Series}
\begin{itemize}
    \item \textbf{Addition}: $\sum a_n x^n + \sum b_n x^n = \sum (a_n + b_n) x^n$
    \item \textbf{Multiplication}: $(\sum a_n x^n)(\sum b_n x^n) = \sum c_n x^n$ where $c_n = \sum_{k=0}^n a_k b_{n-k}$
    \item \textbf{Differentiation}: $\frac{d}{dx}[\sum a_n x^n] = \sum n a_n x^{n-1}$
    \item \textbf{Integration}: $\int [\sum a_n x^n] \, dx = \sum \frac{a_n}{n+1} x^{n+1} + C$
\end{itemize}

\section{Fourier Series}

\subsection{Fourier Coefficients}
For function $f$ with period $2\pi$:
$$a_n = \frac{1}{\pi} \int_{-\pi}^{\pi} f(x) \cos(nx) \, dx$$
$$b_n = \frac{1}{\pi} \int_{-\pi}^{\pi} f(x) \sin(nx) \, dx$$

\subsection{Fourier Series}
$$f(x) = \frac{a_0}{2} + \sum_{n=1}^{\infty} [a_n \cos(nx) + b_n \sin(nx)]$$

\subsection{Complex Form}
$$f(x) = \sum_{n=-\infty}^{\infty} c_n e^{inx}$$
where $c_n = \frac{1}{2\pi} \int_{-\pi}^{\pi} f(x) e^{-inx} \, dx$

\subsection{Parseval's Theorem}
$$\frac{1}{\pi} \int_{-\pi}^{\pi} |f(x)|^2 \, dx = \frac{|a_0|^2}{2} + \sum_{n=1}^{\infty} (|a_n|^2 + |b_n|^2)$$

\section{Partial Differential Equations}

\subsection{Heat Equation}
$$\frac{\partial u}{\partial t} = k \frac{\partial^2 u}{\partial x^2}$$

\subsection{Wave Equation}
$$\frac{\partial^2 u}{\partial t^2} = c^2 \frac{\partial^2 u}{\partial x^2}$$

\subsection{Laplace's Equation}
$$\nabla^2 u = \frac{\partial^2 u}{\partial x^2} + \frac{\partial^2 u}{\partial y^2} + \frac{\partial^2 u}{\partial z^2} = 0$$

\subsection{Method of Separation of Variables}
Assume solution of form $u(x,t) = X(x)T(t)$ and substitute into PDE.

\section{Complex Analysis}

\subsection{Complex Functions}
\begin{definition}
A \textbf{complex function} is a function $f: \mathbb{C} \to \mathbb{C}$.
\end{definition}

\subsection{Cauchy-Riemann Equations}
For $f(z) = u(x,y) + iv(x,y)$ to be differentiable:
$$\frac{\partial u}{\partial x} = \frac{\partial v}{\partial y}, \quad \frac{\partial u}{\partial y} = -\frac{\partial v}{\partial x}$$

\subsection{Cauchy's Integral Theorem}
\begin{theorem}
If $f$ is analytic in simply connected domain $D$ and $C$ is a closed curve in $D$, then:
$$\oint_C f(z) \, dz = 0$$
\end{theorem}

\subsection{Cauchy's Integral Formula}
\begin{theorem}
If $f$ is analytic inside and on simple closed curve $C$, then for any point $a$ inside $C$:
$$f(a) = \frac{1}{2\pi i} \oint_C \frac{f(z)}{z-a} \, dz$$
\end{theorem}

\subsection{Residue Theorem}
\begin{theorem}
$$\oint_C f(z) \, dz = 2\pi i \sum \text{Res}(f, a_k)$$
where the sum is over all isolated singularities inside $C$.
\end{theorem}

\section{Important Theorems}

\subsection{Mean Value Theorem for Integrals}
\begin{theorem}
If $f$ is continuous on $[a,b]$, then there exists $c \in [a,b]$ such that:
$$\int_a^b f(x) \, dx = f(c)(b-a)$$
\end{theorem}

\subsection{Fubini's Theorem}
\begin{theorem}
If $f$ is continuous on rectangle $R = [a,b] \times [c,d]$, then:
$$\iint_R f(x,y) \, dA = \int_a^b \int_c^d f(x,y) \, dy \, dx = \int_c^d \int_a^b f(x,y) \, dx \, dy$$
\end{theorem}

\subsection{Change of Variables}
\begin{theorem}
$$\iint_R f(x,y) \, dA = \iint_S f(x(u,v), y(u,v)) \left|\frac{\partial(x,y)}{\partial(u,v)}\right| \, du \, dv$$
\end{theorem}

\subsection{Implicit Function Theorem}
\begin{theorem}
If $F(x,y) = 0$ and $\frac{\partial F}{\partial y} \neq 0$ at $(a,b)$, then there exists a function $y = f(x)$ such that $F(x,f(x)) = 0$ near $(a,b)$.
\end{theorem}

\section{Applications}

\subsection{Optimization Problems}
\begin{itemize}
    \item Find extrema of functions of several variables
    \item Constrained optimization using Lagrange multipliers
    \item Applications in economics, physics, engineering
\end{itemize}

\subsection{Volume and Surface Area}
\begin{itemize}
    \item Triple integrals for volume calculations
    \item Surface integrals for surface area
    \item Applications in geometry and physics
\end{itemize}

\subsection{Flux and Circulation}
\begin{itemize}
    \item Line integrals for work and circulation
    \item Surface integrals for flux
    \item Applications in fluid dynamics and electromagnetism
\end{itemize}

\subsection{Heat and Wave Propagation}
\begin{itemize}
    \item Fourier series for periodic phenomena
    \item PDEs for modeling physical processes
    \item Applications in physics and engineering
\end{itemize}

\end{document}
