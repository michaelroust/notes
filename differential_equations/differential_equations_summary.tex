\documentclass[11pt]{article}
\usepackage[utf8]{inputenc}
\usepackage{amsmath}
\usepackage{amsfonts}
\usepackage{amssymb}
\usepackage{geometry}
\usepackage{enumitem}
\usepackage{graphicx}
\usepackage{tikz}
\usepackage{pgfplots}
\usepackage{amsthm}
\usepackage{mathtools}

\geometry{margin=1in}

\theoremstyle{definition}
\newtheorem{definition}{Definition}[section]
\newtheorem{theorem}{Theorem}[section]
\newtheorem{lemma}{Lemma}[section]
\newtheorem{corollary}{Corollary}[section]
\newtheorem{example}{Example}[section]
\newtheorem{proposition}{Proposition}[section]

\title{Differential Equations Summary}
\author{Mathematical Notes}
\date{\today}

\begin{document}

\maketitle

\tableofcontents
\newpage

\section{Ordinary Differential Equations}

\subsection{First-Order ODEs}
\begin{definition}
A first-order ODE has the form:
$$\frac{dy}{dx} = f(x, y)$$
\end{definition}

\subsubsection{Separable Equations}
\begin{definition}
A separable equation has the form:
$$\frac{dy}{dx} = g(x)h(y)$$
Solution: $\int \frac{dy}{h(y)} = \int g(x) \, dx + C$
\end{definition}

\subsubsection{Linear Equations}
\begin{definition}
A linear first-order ODE has the form:
$$\frac{dy}{dx} + P(x)y = Q(x)$$
Solution: $y = e^{-\int P(x) \, dx} \left[\int Q(x) e^{\int P(x) \, dx} \, dx + C\right]$
\end{definition}

\subsubsection{Exact Equations}
\begin{definition}
An equation $M(x,y) \, dx + N(x,y) \, dy = 0$ is exact if $\frac{\partial M}{\partial y} = \frac{\partial N}{\partial x}$.
\end{definition}

\subsection{Second-Order Linear ODEs}
\begin{definition}
A second-order linear ODE has the form:
$$a(x)\frac{d^2y}{dx^2} + b(x)\frac{dy}{dx} + c(x)y = f(x)$$
\end{definition}

\subsubsection{Homogeneous Case}
For $f(x) = 0$, the general solution is:
$$y = c_1 y_1(x) + c_2 y_2(x)$$
where $y_1$ and $y_2$ are linearly independent solutions.

\subsubsection{Characteristic Equation}
For constant coefficients $ay'' + by' + cy = 0$:
$$ar^2 + br + c = 0$$
\begin{itemize}
    \item Two real roots: $y = c_1 e^{r_1 x} + c_2 e^{r_2 x}$
    \item One real root: $y = (c_1 + c_2 x) e^{r x}$
    \item Complex roots: $y = e^{\alpha x}(c_1 \cos \beta x + c_2 \sin \beta x)$
\end{itemize}

\subsection{Systems of ODEs}
\begin{definition}
A system of first-order ODEs:
$$\frac{d\mathbf{x}}{dt} = \mathbf{f}(t, \mathbf{x})$$
where $\mathbf{x} = (x_1, \ldots, x_n)^T$.
\end{definition}

\subsubsection{Linear Systems}
For $\frac{d\mathbf{x}}{dt} = A\mathbf{x}$:
$$\mathbf{x}(t) = e^{At}\mathbf{x}_0$$
where $e^{At}$ is the matrix exponential.

% Illustration of phase portraits
\begin{figure}[h]
\centering
\begin{tikzpicture}[scale=1.2]
    \begin{axis}[
        axis lines = center,
        xlabel = $x$,
        ylabel = $y$,
        xmin=-3, xmax=3,
        ymin=-3, ymax=3,
    ]
    
    % Stable node
    \addplot[blue, thick, ->] coordinates {(-2,-2) (-1.5,-1.5)};
    \addplot[blue, thick, ->] coordinates {(2,2) (1.5,1.5)};
    \addplot[blue, thick, ->] coordinates {(-2,2) (-1.5,1.5)};
    \addplot[blue, thick, ->] coordinates {(2,-2) (1.5,-1.5)};
    
    % Saddle point
    \addplot[red, thick, ->] coordinates {(-1,0.5) (-0.5,0.25)};
    \addplot[red, thick, ->] coordinates {(1,-0.5) (0.5,-0.25)};
    \addplot[red, thick, ->] coordinates {(-0.5,-1) (-0.25,-0.5)};
    \addplot[red, thick, ->] coordinates {(0.5,1) (0.25,0.5)};
    
    \node[blue, above] at (0,2.5) {Stable Node};
    \node[red, below] at (0,-2.5) {Saddle Point};
    
    \end{axis}
\end{tikzpicture}
\caption{Phase portraits for linear systems}
\end{figure}

\section{Existence and Uniqueness}

\subsection{Picard-Lindelöf Theorem}
\begin{theorem}
If $f(t,y)$ is continuous and Lipschitz in $y$ on a rectangle $R$, then the IVP:
$$\frac{dy}{dt} = f(t,y), \quad y(t_0) = y_0$$
has a unique solution on some interval containing $t_0$.
\end{theorem}

\subsection{Lipschitz Condition}
\begin{definition}
A function $f(t,y)$ satisfies a Lipschitz condition if:
$$|f(t,y_1) - f(t,y_2)| \leq L|y_1 - y_2|$$
for some constant $L > 0$.
\end{definition}

\section{Stability Theory}

\subsection{Equilibrium Points}
\begin{definition}
An equilibrium point of $\frac{dx}{dt} = f(x)$ is a point $x^*$ such that $f(x^*) = 0$.
\end{definition}

\subsection{Linear Stability Analysis}
\begin{definition}
For a linear system $\frac{dx}{dt} = Ax$, the stability is determined by the eigenvalues of $A$:
\begin{itemize}
    \item All eigenvalues have negative real parts: asymptotically stable
    \item Any eigenvalue has positive real part: unstable
    \item Zero real parts: need further analysis
\end{itemize}
\end{definition}

\subsection{Lyapunov Stability}
\begin{definition}
An equilibrium point $x^*$ is:
\begin{itemize}
    \item \textbf{Stable} if for every $\epsilon > 0$, there exists $\delta > 0$ such that $|x(0) - x^*| < \delta$ implies $|x(t) - x^*| < \epsilon$ for all $t \geq 0$
    \item \textbf{Asymptotically stable} if it's stable and $\lim_{t \to \infty} x(t) = x^*$
\end{itemize}
\end{definition}

\subsection{Lyapunov's Method}
\begin{theorem}
If there exists a Lyapunov function $V(x)$ such that:
\begin{itemize}
    \item $V(x^*) = 0$ and $V(x) > 0$ for $x \neq x^*$
    \item $\dot{V}(x) \leq 0$ for all $x$
\end{itemize}
then $x^*$ is stable. If $\dot{V}(x) < 0$ for $x \neq x^*$, then $x^*$ is asymptotically stable.
\end{theorem}

\section{Partial Differential Equations}

\subsection{Classification}
\begin{definition}
A second-order linear PDE in two variables:
$$A \frac{\partial^2 u}{\partial x^2} + B \frac{\partial^2 u}{\partial x \partial y} + C \frac{\partial^2 u}{\partial y^2} + D \frac{\partial u}{\partial x} + E \frac{\partial u}{\partial y} + Fu = G$$
is classified by the discriminant $\Delta = B^2 - 4AC$:
\begin{itemize}
    \item $\Delta > 0$: Hyperbolic
    \item $\Delta = 0$: Parabolic
    \item $\Delta < 0$: Elliptic
\end{itemize}
\end{definition}

\subsection{Wave Equation}
\begin{definition}
The wave equation:
$$\frac{\partial^2 u}{\partial t^2} = c^2 \frac{\partial^2 u}{\partial x^2}$$
General solution: $u(x,t) = f(x-ct) + g(x+ct)$
\end{definition}

\subsection{Heat Equation}
\begin{definition}
The heat equation:
$$\frac{\partial u}{\partial t} = \alpha \frac{\partial^2 u}{\partial x^2}$$
Solution by separation of variables: $u(x,t) = X(x)T(t)$
\end{definition}

\subsection{Laplace's Equation}
\begin{definition}
Laplace's equation:
$$\frac{\partial^2 u}{\partial x^2} + \frac{\partial^2 u}{\partial y^2} = 0$$
Solutions are harmonic functions.
\end{definition}

\section{Method of Characteristics}

\subsection{First-Order PDEs}
\begin{definition}
For the PDE $a(x,y,u) \frac{\partial u}{\partial x} + b(x,y,u) \frac{\partial u}{\partial y} = c(x,y,u)$, the characteristic equations are:
$$\frac{dx}{ds} = a, \quad \frac{dy}{ds} = b, \quad \frac{du}{ds} = c$$
\end{definition}

\section{Green's Functions}

\subsection{Definition}
\begin{definition}
A Green's function $G(x,\xi)$ for the operator $L$ satisfies:
$$LG(x,\xi) = \delta(x-\xi)$$
where $\delta$ is the Dirac delta function.
\end{definition}

\subsection{Solution Representation}
\begin{theorem}
If $Lu = f$ with homogeneous boundary conditions, then:
$$u(x) = \int G(x,\xi) f(\xi) \, d\xi$$
\end{theorem}

\section{Fourier Methods}

\subsection{Fourier Series}
\begin{definition}
For a periodic function $f(x)$ with period $2L$:
$$f(x) = \frac{a_0}{2} + \sum_{n=1}^{\infty} \left(a_n \cos \frac{n\pi x}{L} + b_n \sin \frac{n\pi x}{L}\right)$$
where:
$$a_n = \frac{1}{L} \int_{-L}^L f(x) \cos \frac{n\pi x}{L} \, dx$$
$$b_n = \frac{1}{L} \int_{-L}^L f(x) \sin \frac{n\pi x}{L} \, dx$$
\end{definition}

\subsection{Fourier Transform}
\begin{definition}
The Fourier transform of $f(x)$ is:
$$\hat{f}(\xi) = \int_{-\infty}^{\infty} f(x) e^{-2\pi i \xi x} \, dx$$
Inverse transform:
$$f(x) = \int_{-\infty}^{\infty} \hat{f}(\xi) e^{2\pi i \xi x} \, d\xi$$
\end{definition}

\section{Applications}

\subsection{Physics}
Differential equations model:
\begin{itemize}
    \item Classical mechanics (Newton's laws)
    \item Electromagnetism (Maxwell's equations)
    \item Quantum mechanics (Schrödinger equation)
    \item Fluid dynamics (Navier-Stokes equations)
\end{itemize}

\subsection{Biology}
Applications include:
\begin{itemize}
    \item Population dynamics
    \item Epidemiology
    \item Chemical kinetics
    \item Neural networks
\end{itemize}

\subsection{Engineering}
Used in:
\begin{itemize}
    \item Control systems
    \item Signal processing
    \item Heat transfer
    \item Structural analysis
\end{itemize}

\section{Important Theorems}

\subsection{Existence and Uniqueness for Systems}
\begin{theorem}
If $\mathbf{f}(t,\mathbf{x})$ is continuous and satisfies a Lipschitz condition in $\mathbf{x}$ on a domain $D$, then the IVP $\frac{d\mathbf{x}}{dt} = \mathbf{f}(t,\mathbf{x})$, $\mathbf{x}(t_0) = \mathbf{x}_0$ has a unique solution.
\end{theorem}

\subsection{Sturm-Liouville Theory}
\begin{theorem}
For the Sturm-Liouville problem:
$$-\frac{d}{dx}\left[p(x)\frac{dy}{dx}\right] + q(x)y = \lambda w(x)y$$
with appropriate boundary conditions, the eigenvalues are real and the eigenfunctions are orthogonal.
\end{theorem}

\subsection{Maximum Principle}
\begin{theorem}
For Laplace's equation in a bounded domain, the maximum and minimum values occur on the boundary.
\end{theorem}

\section{Numerical Methods}

\subsection{Finite Differences}
\begin{definition}
Finite difference approximations:
\begin{itemize}
    \item Forward: $f'(x) \approx \frac{f(x+h) - f(x)}{h}$
    \item Backward: $f'(x) \approx \frac{f(x) - f(x-h)}{h}$
    \item Central: $f'(x) \approx \frac{f(x+h) - f(x-h)}{2h}$
\end{itemize}
\end{definition}

\subsection{Finite Elements}
\begin{definition}
The finite element method approximates the solution by piecewise polynomial functions on a mesh.
\end{definition}

\end{document}
