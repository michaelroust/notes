\documentclass[11pt]{article}
\usepackage[utf8]{inputenc}
\usepackage{amsmath}
\usepackage{amsfonts}
\usepackage{amssymb}
\usepackage{geometry}
\usepackage{enumitem}
\usepackage{graphicx}
\usepackage{tikz}
\usepackage{pgfplots}
\usepackage{amsthm}
\usepackage{mathtools}

\geometry{margin=1in}

\theoremstyle{definition}
\newtheorem{definition}{Definition}[section]
\newtheorem{theorem}{Theorem}[section]
\newtheorem{lemma}{Lemma}[section]
\newtheorem{corollary}{Corollary}[section]
\newtheorem{example}{Example}[section]
\newtheorem{proposition}{Proposition}[section]

\title{Relativity and Quantum Physics Summary}
\author{Mathematical Notes}
\date{\today}

\begin{document}

\maketitle

\tableofcontents
\newpage

\section{Special Relativity}

\subsection{Postulates}
\begin{enumerate}
    \item \textbf{Principle of Relativity}: The laws of physics are the same in all inertial reference frames.
    \item \textbf{Constancy of Light Speed}: The speed of light in vacuum is constant ($c = 2.998 \times 10^8$ m/s) in all inertial frames.
\end{enumerate}

\subsection{Lorentz Transformations}
For frames moving with relative velocity $v$ along $x$-axis:
\begin{align}
    x' &= \gamma(x - vt) \\
    t' &= \gamma\left(t - \frac{vx}{c^2}\right) \\
    y' &= y \\
    z' &= z
\end{align}
where $\gamma = \frac{1}{\sqrt{1-\beta^2}}$ and $\beta = \frac{v}{c}$.

\subsection{Time Dilation and Length Contraction}
\begin{theorem}[Time Dilation]
$$\Delta t = \gamma \Delta t_0$$
where $\Delta t_0$ is the proper time (time in the rest frame).
\end{theorem}

\begin{theorem}[Length Contraction]
$$L = \frac{L_0}{\gamma}$$
where $L_0$ is the proper length (length in the rest frame).
\end{theorem}

\subsection{Relativistic Velocity Addition}
For velocities $u$ and $v$ along the same direction:
$$u' = \frac{u + v}{1 + \frac{uv}{c^2}}$$

\subsection{Relativistic Energy and Momentum}
\begin{itemize}
    \item \textbf{Relativistic momentum}: $\vec{p} = \gamma m \vec{v}$
    \item \textbf{Total energy}: $E = \gamma mc^2$
    \item \textbf{Rest energy}: $E_0 = mc^2$
    \item \textbf{Kinetic energy}: $K = (\gamma - 1)mc^2$
    \item \textbf{Energy-momentum relation}: $E^2 = (pc)^2 + (mc^2)^2$
\end{itemize}

\subsection{Four-Vectors}
\begin{definition}
A \textbf{four-vector} is a quantity that transforms under Lorentz transformations like $(ct, x, y, z)$.
\end{definition}

\begin{itemize}
    \item \textbf{Position four-vector}: $x^\mu = (ct, \vec{r})$
    \item \textbf{Energy-momentum four-vector}: $p^\mu = (E/c, \vec{p})$
    \item \textbf{Four-velocity}: $u^\mu = \gamma(c, \vec{v})$
    \item \textbf{Four-force}: $F^\mu = \gamma(\vec{F} \cdot \vec{v}/c, \vec{F})$
\end{itemize}

\section{General Relativity}

\subsection{Equivalence Principle}
\begin{definition}
The \textbf{weak equivalence principle} states that the gravitational and inertial masses are equivalent.
\end{definition}

\begin{definition}
The \textbf{strong equivalence principle} states that the effects of gravity are locally indistinguishable from acceleration.
\end{definition}

\subsection{Spacetime Curvature}
\begin{definition}
\textbf{Spacetime} is a four-dimensional manifold with metric tensor $g_{\mu\nu}$.
\end{definition}

The line element is:
$$ds^2 = g_{\mu\nu} dx^\mu dx^\nu$$

\subsection{Einstein Field Equations}
\begin{theorem}[Einstein Field Equations]
$$G_{\mu\nu} + \Lambda g_{\mu\nu} = \frac{8\pi G}{c^4} T_{\mu\nu}$$
where:
\begin{itemize}
    \item $G_{\mu\nu}$ is the Einstein tensor
    \item $\Lambda$ is the cosmological constant
    \item $T_{\mu\nu}$ is the stress-energy tensor
\end{itemize}
\end{theorem}

\subsection{Schwarzschild Metric}
For a spherically symmetric mass $M$:
$$ds^2 = -\left(1 - \frac{2GM}{c^2r}\right)c^2dt^2 + \left(1 - \frac{2GM}{c^2r}\right)^{-1}dr^2 + r^2(d\theta^2 + \sin^2\theta d\phi^2)$$

\subsection{Black Holes}
\begin{definition}
The \textbf{Schwarzschild radius} is $r_s = \frac{2GM}{c^2}$.
\end{definition}

\begin{definition}
An \textbf{event horizon} is a boundary beyond which events cannot affect an outside observer.
\end{definition}

\section{Quantum Mechanics Fundamentals}

\subsection{Wave-Particle Duality}
\begin{definition}
\textbf{de Broglie wavelength}: $\lambda = \frac{h}{p}$ where $h$ is Planck's constant.
\end{definition}

\begin{definition}
\textbf{Photoelectric effect}: $E_k = h\nu - \phi$ where $\phi$ is the work function.
\end{definition}

\subsection{Uncertainty Principle}
\begin{theorem}[Heisenberg Uncertainty Principle]
$$\Delta x \Delta p \geq \frac{\hbar}{2}$$
$$\Delta E \Delta t \geq \frac{\hbar}{2}$$
where $\hbar = \frac{h}{2\pi}$.
\end{theorem}

\subsection{Schrödinger Equation}
\begin{theorem}[Time-Dependent Schrödinger Equation]
$$i\hbar \frac{\partial \psi}{\partial t} = \hat{H}\psi$$
where $\hat{H} = -\frac{\hbar^2}{2m}\nabla^2 + V(\vec{r},t)$ is the Hamiltonian operator.
\end{theorem}

\begin{theorem}[Time-Independent Schrödinger Equation]
$$\hat{H}\psi = E\psi$$
for stationary states with energy $E$.
\end{theorem}

\subsection{Quantum States and Operators}
\begin{definition}
A \textbf{quantum state} is represented by a wave function $\psi(\vec{r},t)$ normalized as $\int |\psi|^2 d^3r = 1$.
\end{definition}

\begin{definition}
An \textbf{observable} is represented by a Hermitian operator $\hat{A}$ with eigenvalues $a_n$ and eigenstates $|n\rangle$.
\end{definition}

\subsection{Measurement and Collapse}
\begin{definition}
Upon measurement of observable $\hat{A}$, the state collapses to an eigenstate $|n\rangle$ with probability $|\langle n|\psi\rangle|^2$.
\end{definition}

\section{Quantum Systems}

\subsection{Particle in a Box}
For a particle in a one-dimensional box of length $L$:
$$\psi_n(x) = \sqrt{\frac{2}{L}} \sin\left(\frac{n\pi x}{L}\right)$$
$$E_n = \frac{n^2\pi^2\hbar^2}{2mL^2}$$

\subsection{Harmonic Oscillator}
\begin{definition}
The \textbf{quantum harmonic oscillator} has energy levels:
$$E_n = \hbar\omega\left(n + \frac{1}{2}\right)$$
where $n = 0, 1, 2, \ldots$
\end{definition}

The wave functions involve Hermite polynomials:
$$\psi_n(x) = \left(\frac{m\omega}{\pi\hbar}\right)^{1/4} \frac{1}{\sqrt{2^n n!}} H_n\left(\sqrt{\frac{m\omega}{\hbar}}x\right) e^{-\frac{m\omega x^2}{2\hbar}}$$

\subsection{Hydrogen Atom}
\begin{theorem}[Bohr Model]
$$r_n = \frac{n^2\hbar^2}{me^2} \frac{4\pi\epsilon_0}{e^2}$$
$$E_n = -\frac{me^4}{2\hbar^2(4\pi\epsilon_0)^2} \frac{1}{n^2}$$
\end{theorem}

\begin{theorem}[Quantum Mechanical Solution]
The wave function is $\psi_{nlm}(r,\theta,\phi) = R_{nl}(r)Y_l^m(\theta,\phi)$ where:
\begin{itemize}
    \item $n$ is the principal quantum number
    \item $l$ is the orbital angular momentum quantum number
    \item $m$ is the magnetic quantum number
\end{itemize}
\end{theorem}

\section{Angular Momentum}

\subsection{Orbital Angular Momentum}
\begin{definition}
The \textbf{orbital angular momentum operator} is $\hat{\vec{L}} = \hat{\vec{r}} \times \hat{\vec{p}}$.
\end{definition}

\begin{itemize}
    \item $\hat{L}^2 |l,m\rangle = l(l+1)\hbar^2 |l,m\rangle$
    \item $\hat{L}_z |l,m\rangle = m\hbar |l,m\rangle$
    \item $l = 0, 1, 2, \ldots$ and $m = -l, -l+1, \ldots, l-1, l$
\end{itemize}

\subsection{Spin}
\begin{definition}
\textbf{Spin} is an intrinsic angular momentum property of particles.
\end{definition}

For spin-1/2 particles:
$$\hat{S}^2 |s,m_s\rangle = s(s+1)\hbar^2 |s,m_s\rangle = \frac{3}{4}\hbar^2 |s,m_s\rangle$$
$$\hat{S}_z |s,m_s\rangle = m_s\hbar |s,m_s\rangle$$
where $m_s = \pm\frac{1}{2}$.

\subsection{Pauli Matrices}
$$\sigma_x = \begin{pmatrix} 0 & 1 \\ 1 & 0 \end{pmatrix}, \quad \sigma_y = \begin{pmatrix} 0 & -i \\ i & 0 \end{pmatrix}, \quad \sigma_z = \begin{pmatrix} 1 & 0 \\ 0 & -1 \end{pmatrix}$$

\section{Identical Particles}

\subsection{Symmetrization Postulate}
\begin{definition}
For identical particles, the wave function must be either:
\begin{itemize}
    \item \textbf{Symmetric} (bosons): $\psi(1,2) = \psi(2,1)$
    \item \textbf{Antisymmetric} (fermions): $\psi(1,2) = -\psi(2,1)$
\end{itemize}
\end{definition}

\subsection{Pauli Exclusion Principle}
\begin{theorem}
No two identical fermions can occupy the same quantum state.
\end{theorem}

\subsection{Exchange Symmetry}
\begin{itemize}
    \item \textbf{Bosons} (integer spin): Follow Bose-Einstein statistics
    \item \textbf{Fermions} (half-integer spin): Follow Fermi-Dirac statistics
\end{itemize}

\section{Quantum Field Theory}

\subsection{Field Quantization}
\begin{definition}
A \textbf{quantum field} is an operator-valued function $\hat{\phi}(x)$ that creates and annihilates particles.
\end{definition}

\subsection{Creation and Annihilation Operators}
\begin{itemize}
    \item $\hat{a}^\dagger$ creates a particle
    \item $\hat{a}$ annihilates a particle
    \item $[\hat{a}, \hat{a}^\dagger] = 1$ (bosons)
    \item $\{\hat{a}, \hat{a}^\dagger\} = 1$ (fermions)
\end{itemize}

\subsection{Feynman Diagrams}
\begin{definition}
\textbf{Feynman diagrams} are graphical representations of particle interactions in quantum field theory.
\end{definition}

\section{Relativistic Quantum Mechanics}

\subsection{Klein-Gordon Equation}
For spin-0 particles:
$$\left(\frac{1}{c^2}\frac{\partial^2}{\partial t^2} - \nabla^2 + \frac{m^2c^2}{\hbar^2}\right)\psi = 0$$

\subsection{Dirac Equation}
For spin-1/2 particles:
$$(i\gamma^\mu\partial_\mu - m)\psi = 0$$
where $\gamma^\mu$ are the Dirac matrices.

\subsection{Antimatter}
\begin{definition}
\textbf{Antiparticles} have the same mass but opposite charge and quantum numbers as their corresponding particles.
\end{definition}

\section{Quantum Entanglement}

\subsection{Bell States}
The maximally entangled two-qubit states:
\begin{align}
|\Phi^+\rangle &= \frac{1}{\sqrt{2}}(|00\rangle + |11\rangle) \\
|\Phi^-\rangle &= \frac{1}{\sqrt{2}}(|00\rangle - |11\rangle) \\
|\Psi^+\rangle &= \frac{1}{\sqrt{2}}(|01\rangle + |10\rangle) \\
|\Psi^-\rangle &= \frac{1}{\sqrt{2}}(|01\rangle - |10\rangle)
\end{align}

\subsection{Bell's Theorem}
\begin{theorem}
No local hidden variable theory can reproduce all the predictions of quantum mechanics.
\end{theorem}

\subsection{EPR Paradox}
The Einstein-Podolsky-Rosen paradox questions the completeness of quantum mechanics and introduces the concept of "spooky action at a distance."

\section{Applications}

\subsection{Quantum Computing}
\begin{itemize}
    \item Quantum gates and circuits
    \item Quantum algorithms (Shor's, Grover's)
    \item Quantum error correction
\end{itemize}

\subsection{Quantum Cryptography}
\begin{itemize}
    \item Quantum key distribution (BB84 protocol)
    \item Quantum teleportation
    \item Quantum secure communication
\end{itemize}

\subsection{Quantum Sensing}
\begin{itemize}
    \item Quantum metrology
    \item Gravitational wave detection
    \item Quantum magnetometry
\end{itemize}

\subsection{Cosmology}
\begin{itemize}
    \item Big Bang theory
    \item Cosmic microwave background
    \item Dark matter and dark energy
    \item Inflation theory
\end{itemize}

\section{Important Constants}

\begin{itemize}
    \item Speed of light: $c = 2.998 \times 10^8$ m/s
    \item Planck constant: $h = 6.626 \times 10^{-34}$ J·s
    \item Reduced Planck constant: $\hbar = 1.055 \times 10^{-34}$ J·s
    \item Gravitational constant: $G = 6.674 \times 10^{-11}$ N·m²/kg²
    \item Electron charge: $e = 1.602 \times 10^{-19}$ C
    \item Electron mass: $m_e = 9.109 \times 10^{-31}$ kg
    \item Proton mass: $m_p = 1.673 \times 10^{-27}$ kg
    \item Fine structure constant: $\alpha = \frac{e^2}{4\pi\epsilon_0\hbar c} \approx \frac{1}{137}$
    \item Bohr radius: $a_0 = \frac{4\pi\epsilon_0\hbar^2}{me^2} = 5.292 \times 10^{-11}$ m
    \item Rydberg constant: $R_\infty = \frac{me^4}{8\epsilon_0^2h^3c} = 1.097 \times 10^7$ m$^{-1}$
\end{itemize}

\section{Key Theorems}

\subsection{No-Cloning Theorem}
\begin{theorem}
It is impossible to create an identical copy of an arbitrary unknown quantum state.
\end{theorem}

\subsection{Unitarity}
\begin{theorem}
Quantum evolution is unitary: $\hat{U}^\dagger\hat{U} = \hat{I}$.
\end{theorem}

\subsection{Conservation Laws}
\begin{itemize}
    \item Energy conservation
    \item Momentum conservation
    \item Angular momentum conservation
    \item Charge conservation
    \item Baryon and lepton number conservation
\end{itemize}

\section{Modern Developments}

\subsection{Quantum Field Theory}
\begin{itemize}
    \item Standard Model of particle physics
    \item Quantum electrodynamics (QED)
    \item Quantum chromodynamics (QCD)
    \item Electroweak theory
\end{itemize}

\subsection{Quantum Gravity}
\begin{itemize}
    \item String theory
    \item Loop quantum gravity
    \item Causal dynamical triangulation
    \item Asymptotic safety
\end{itemize}

\subsection{Quantum Information}
\begin{itemize}
    \item Quantum error correction
    \item Quantum communication
    \item Quantum simulation
    \item Quantum machine learning
\end{itemize}

\end{document}
