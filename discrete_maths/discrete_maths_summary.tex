\documentclass[11pt]{article}
\usepackage[utf8]{inputenc}
\usepackage{amsmath}
\usepackage{amsfonts}
\usepackage{amssymb}
\usepackage{geometry}
\usepackage{enumitem}
\usepackage{graphicx}
\usepackage{tikz}
\usepackage{pgfplots}
\usepackage{amsthm}
\usepackage{mathtools}

\geometry{margin=1in}

\theoremstyle{definition}
\newtheorem{definition}{Definition}[section]
\newtheorem{theorem}{Theorem}[section]
\newtheorem{lemma}{Lemma}[section]
\newtheorem{corollary}{Corollary}[section]
\newtheorem{example}{Example}[section]

\title{Discrete Mathematics Summary}
\author{Mathematical Notes}
\date{\today}

\begin{document}

\maketitle

\tableofcontents
\newpage

\section{Logic and Proofs}

\subsection{Propositional Logic}

\begin{definition}
A \textbf{proposition} is a declarative sentence that is either true or false, but not both.
\end{definition}

\subsection{Logical Connectives}
\begin{itemize}
    \item \textbf{Negation}: $\neg p$ (not $p$)
    \item \textbf{Conjunction}: $p \land q$ ($p$ and $q$)
    \item \textbf{Disjunction}: $p \lor q$ ($p$ or $q$)
    \item \textbf{Implication}: $p \rightarrow q$ (if $p$ then $q$)
    \item \textbf{Biconditional}: $p \leftrightarrow q$ ($p$ if and only if $q$)
\end{itemize}

\subsection{Truth Tables}
\begin{center}
\begin{tabular}{|c|c|c|c|c|c|}
\hline
$p$ & $q$ & $\neg p$ & $p \land q$ & $p \lor q$ & $p \rightarrow q$ \\
\hline
T & T & F & T & T & T \\
T & F & F & F & T & F \\
F & T & T & F & T & T \\
F & F & T & F & F & T \\
\hline
\end{tabular}
\end{center}

\subsection{Logical Equivalences}
\begin{itemize}
    \item \textbf{Double Negation}: $\neg(\neg p) \equiv p$
    \item \textbf{De Morgan's Laws}: 
        \begin{itemize}
            \item $\neg(p \land q) \equiv \neg p \lor \neg q$
            \item $\neg(p \lor q) \equiv \neg p \land \neg q$
        \end{itemize}
    \item \textbf{Commutative Laws}: $p \land q \equiv q \land p$, $p \lor q \equiv q \lor p$
    \item \textbf{Associative Laws}: $(p \land q) \land r \equiv p \land (q \land r)$
    \item \textbf{Distributive Laws}: 
        \begin{itemize}
            \item $p \land (q \lor r) \equiv (p \land q) \lor (p \land r)$
            \item $p \lor (q \land r) \equiv (p \lor q) \land (p \lor r)$
        \end{itemize}
    \item \textbf{Implication}: $p \rightarrow q \equiv \neg p \lor q$
    \item \textbf{Contrapositive}: $p \rightarrow q \equiv \neg q \rightarrow \neg p$
\end{itemize}

\subsection{Predicate Logic}
\begin{definition}
A \textbf{predicate} is a statement involving variables that becomes a proposition when specific values are substituted for the variables.
\end{definition}

\subsection{Quantifiers}
\begin{itemize}
    \item \textbf{Universal Quantifier}: $\forall x P(x)$ (for all $x$, $P(x)$)
    \item \textbf{Existential Quantifier}: $\exists x P(x)$ (there exists an $x$ such that $P(x)$)
\end{itemize}

\subsection{Methods of Proof}
\begin{itemize}
    \item \textbf{Direct Proof}: Assume $p$ is true, show $q$ is true
    \item \textbf{Proof by Contraposition}: Prove $\neg q \rightarrow \neg p$
    \item \textbf{Proof by Contradiction}: Assume $\neg(p \rightarrow q)$, derive a contradiction
    \item \textbf{Proof by Cases}: Consider all possible cases
    \item \textbf{Mathematical Induction}: 
        \begin{enumerate}
            \item Base case: Show $P(1)$ is true
            \item Inductive step: Show $P(k) \rightarrow P(k+1)$ for all $k \geq 1$
        \end{enumerate}
\end{itemize}

\section{Sets}

\subsection{Basic Definitions}
\begin{definition}
A \textbf{set} is an unordered collection of distinct objects called elements.
\end{definition}

\subsection{Set Operations}
\begin{itemize}
    \item \textbf{Union}: $A \cup B = \{x : x \in A \text{ or } x \in B\}$
    \item \textbf{Intersection}: $A \cap B = \{x : x \in A \text{ and } x \in B\}$
    \item \textbf{Complement}: $\overline{A} = \{x : x \notin A\}$
    \item \textbf{Difference}: $A - B = \{x : x \in A \text{ and } x \notin B\}$
    \item \textbf{Symmetric Difference}: $A \triangle B = (A - B) \cup (B - A)$
\end{itemize}

\subsection{Set Identities}
\begin{itemize}
    \item \textbf{Commutative Laws}: $A \cup B = B \cup A$, $A \cap B = B \cap A$
    \item \textbf{Associative Laws}: $(A \cup B) \cup C = A \cup (B \cup C)$
    \item \textbf{Distributive Laws}: 
        \begin{itemize}
            \item $A \cup (B \cap C) = (A \cup B) \cap (A \cup C)$
            \item $A \cap (B \cup C) = (A \cap B) \cup (A \cap C)$
        \end{itemize}
    \item \textbf{De Morgan's Laws}: 
        \begin{itemize}
            \item $\overline{A \cup B} = \overline{A} \cap \overline{B}$
            \item $\overline{A \cap B} = \overline{A} \cup \overline{B}$
        \end{itemize}
\end{itemize}

\subsection{Cardinality}
\begin{definition}
The \textbf{cardinality} of a set $A$, denoted $|A|$, is the number of elements in $A$.
\end{definition}

\subsection{Power Set}
\begin{definition}
The \textbf{power set} of a set $S$, denoted $\mathcal{P}(S)$, is the set of all subsets of $S$.
\end{definition}

\begin{theorem}
If $|S| = n$, then $|\mathcal{P}(S)| = 2^n$.
\end{theorem}

\section{Functions}

\subsection{Basic Definitions}
\begin{definition}
A \textbf{function} $f$ from set $A$ to set $B$ is a relation that assigns to each element $a \in A$ exactly one element $b \in B$. We write $f: A \rightarrow B$.
\end{definition}

\subsection{Types of Functions}
\begin{itemize}
    \item \textbf{One-to-one (Injective)}: $f(a_1) = f(a_2) \Rightarrow a_1 = a_2$
    \item \textbf{Onto (Surjective)}: For every $b \in B$, there exists $a \in A$ such that $f(a) = b$
    \item \textbf{Bijective}: Both one-to-one and onto
\end{itemize}

\subsection{Composition and Inverse}
\begin{itemize}
    \item \textbf{Composition}: $(g \circ f)(x) = g(f(x))$
    \item \textbf{Inverse}: $f^{-1}(y) = x$ if and only if $f(x) = y$
\end{itemize}

\section{Relations}

\subsection{Basic Definitions}
\begin{definition}
A \textbf{relation} $R$ from set $A$ to set $B$ is a subset of $A \times B$.
\end{definition}

\subsection{Properties of Relations}
For a relation $R$ on set $A$:
\begin{itemize}
    \item \textbf{Reflexive}: $(a,a) \in R$ for all $a \in A$
    \item \textbf{Symmetric}: $(a,b) \in R \Rightarrow (b,a) \in R$
    \item \textbf{Antisymmetric}: $(a,b) \in R \land (b,a) \in R \Rightarrow a = b$
    \item \textbf{Transitive}: $(a,b) \in R \land (b,c) \in R \Rightarrow (a,c) \in R$
\end{itemize}

\subsection{Equivalence Relations}
\begin{definition}
An \textbf{equivalence relation} is a relation that is reflexive, symmetric, and transitive.
\end{definition}

\subsection{Partial Orders}
\begin{definition}
A \textbf{partial order} is a relation that is reflexive, antisymmetric, and transitive.
\end{definition}

\section{Combinatorics}

\subsection{Basic Counting Principles}
\begin{itemize}
    \item \textbf{Sum Rule}: If task can be done in $m$ ways and another in $n$ ways, then one or the other can be done in $m + n$ ways
    \item \textbf{Product Rule}: If task can be done in $m$ ways and another in $n$ ways, then both can be done in $m \times n$ ways
\end{itemize}

\subsection{Permutations}
\begin{definition}
A \textbf{permutation} is an ordered arrangement of objects.
\end{definition}

\begin{itemize}
    \item \textbf{Permutations of $n$ objects}: $P(n,n) = n!$
    \item \textbf{Permutations of $r$ objects from $n$}: $P(n,r) = \frac{n!}{(n-r)!}$
\end{itemize}

\subsection{Combinations}
\begin{definition}
A \textbf{combination} is an unordered selection of objects.
\end{definition}

\begin{itemize}
    \item \textbf{Combinations of $r$ objects from $n$}: $C(n,r) = \binom{n}{r} = \frac{n!}{r!(n-r)!}$
\end{itemize}

\subsection{Binomial Theorem}
\begin{theorem}
$$(x + y)^n = \sum_{k=0}^{n} \binom{n}{k} x^{n-k} y^k$$
\end{theorem}

\subsection{Pigeonhole Principle}
\begin{theorem}
If $n$ objects are placed into $m$ boxes where $n > m$, then at least one box contains more than one object.
\end{theorem}

\section{Graph Theory}

\subsection{Basic Definitions}
\begin{definition}
A \textbf{graph} $G = (V, E)$ consists of a set $V$ of vertices and a set $E$ of edges.
\end{definition}

\subsection{Types of Graphs}
\begin{itemize}
    \item \textbf{Simple Graph}: No loops or multiple edges
    \item \textbf{Multigraph}: May have multiple edges
    \item \textbf{Pseudograph}: May have loops and multiple edges
    \item \textbf{Directed Graph}: Edges have direction
    \item \textbf{Complete Graph}: Every pair of vertices is connected
    \item \textbf{Bipartite Graph}: Vertices can be partitioned into two sets with no edges within each set
\end{itemize}

\subsection{Graph Terminology}
\begin{itemize}
    \item \textbf{Degree}: Number of edges incident to a vertex
    \item \textbf{Path}: Sequence of vertices connected by edges
    \item \textbf{Circuit}: Path that starts and ends at the same vertex
    \item \textbf{Connected}: Path exists between any two vertices
    \item \textbf{Tree}: Connected graph with no circuits
\end{itemize}

\subsection{Handshaking Theorem}
\begin{theorem}
The sum of the degrees of all vertices in a graph equals twice the number of edges.
$$\sum_{v \in V} \deg(v) = 2|E|$$
\end{theorem}

\subsection{Euler and Hamiltonian Paths}
\begin{itemize}
    \item \textbf{Euler Path}: Uses every edge exactly once
    \item \textbf{Euler Circuit}: Euler path that starts and ends at the same vertex
    \item \textbf{Hamiltonian Path}: Visits every vertex exactly once
    \item \textbf{Hamiltonian Circuit}: Hamiltonian path that starts and ends at the same vertex
\end{itemize}

\subsection{Planar Graphs}
\begin{definition}
A graph is \textbf{planar} if it can be drawn in the plane without edge crossings.
\end{definition}

\begin{theorem}[Euler's Formula]
For a connected planar graph with $V$ vertices, $E$ edges, and $F$ faces:
$$V - E + F = 2$$
\end{theorem}

\section{Number Theory}

\subsection{Divisibility}
\begin{definition}
An integer $a$ \textbf{divides} an integer $b$ (written $a | b$) if there exists an integer $c$ such that $b = ac$.
\end{definition}

\subsection{Properties of Divisibility}
\begin{itemize}
    \item If $a | b$ and $b | c$, then $a | c$
    \item If $a | b$ and $a | c$, then $a | (b + c)$
    \item If $a | b$, then $a | bc$ for any integer $c$
\end{itemize}

\subsection{Division Algorithm}
\begin{theorem}
For integers $a$ and $b$ with $b > 0$, there exist unique integers $q$ and $r$ such that:
$$a = bq + r \quad \text{where } 0 \leq r < b$$
\end{theorem}

\subsection{Greatest Common Divisor}
\begin{definition}
The \textbf{greatest common divisor} of integers $a$ and $b$, denoted $\gcd(a,b)$, is the largest integer that divides both $a$ and $b$.
\end{definition}

\subsection{Euclidean Algorithm}
To find $\gcd(a,b)$:
\begin{enumerate}
    \item If $b = 0$, then $\gcd(a,b) = a$
    \item Otherwise, $\gcd(a,b) = \gcd(b, a \bmod b)$
\end{enumerate}

\subsection{Prime Numbers}
\begin{definition}
A \textbf{prime number} is an integer greater than 1 whose only positive divisors are 1 and itself.
\end{definition}

\subsection{Fundamental Theorem of Arithmetic}
\begin{theorem}
Every integer greater than 1 can be expressed uniquely as a product of primes.
\end{theorem}

\subsection{Congruence}
\begin{definition}
Integers $a$ and $b$ are \textbf{congruent modulo $m$} (written $a \equiv b \pmod{m}$) if $m | (a - b)$.
\end{definition}

\subsection{Properties of Congruence}
\begin{itemize}
    \item $a \equiv a \pmod{m}$ (reflexive)
    \item $a \equiv b \pmod{m} \Rightarrow b \equiv a \pmod{m}$ (symmetric)
    \item $a \equiv b \pmod{m}$ and $b \equiv c \pmod{m} \Rightarrow a \equiv c \pmod{m}$ (transitive)
\end{itemize}

\section{Recurrence Relations}

\subsection{Definition}
\begin{definition}
A \textbf{recurrence relation} is an equation that defines a sequence recursively.
\end{definition}

\subsection{Linear Homogeneous Recurrence Relations}
A recurrence relation of the form:
$$a_n = c_1 a_{n-1} + c_2 a_{n-2} + \cdots + c_k a_{n-k}$$
where $c_1, c_2, \ldots, c_k$ are constants.

\subsection{Solving Linear Homogeneous Recurrence Relations}
\begin{enumerate}
    \item Find the characteristic equation: $r^k - c_1 r^{k-1} - c_2 r^{k-2} - \cdots - c_k = 0$
    \item Find the roots $r_1, r_2, \ldots, r_k$
    \item If all roots are distinct: $a_n = \alpha_1 r_1^n + \alpha_2 r_2^n + \cdots + \alpha_k r_k^n$
    \item If root $r$ has multiplicity $m$: include terms $\alpha_1 r^n, \alpha_2 n r^n, \ldots, \alpha_m n^{m-1} r^n$
\end{enumerate}

\subsection{Common Recurrence Relations}
\begin{itemize}
    \item \textbf{Fibonacci}: $F_n = F_{n-1} + F_{n-2}$ with $F_0 = 0, F_1 = 1$
    \item \textbf{Geometric}: $a_n = r a_{n-1}$ with solution $a_n = a_0 r^n$
    \item \textbf{Arithmetic}: $a_n = a_{n-1} + d$ with solution $a_n = a_0 + nd$
\end{itemize}

\section{Generating Functions}

\subsection{Definition}
\begin{definition}
The \textbf{generating function} for sequence $\{a_n\}$ is:
$$G(x) = \sum_{n=0}^{\infty} a_n x^n$$
\end{definition}

\subsection{Common Generating Functions}
\begin{itemize}
    \item $\frac{1}{1-x} = \sum_{n=0}^{\infty} x^n$ for $|x| < 1$
    \item $\frac{1}{1-ax} = \sum_{n=0}^{\infty} a^n x^n$ for $|ax| < 1$
    \item $(1+x)^n = \sum_{k=0}^{n} \binom{n}{k} x^k$ (binomial theorem)
    \item $e^x = \sum_{n=0}^{\infty} \frac{x^n}{n!}$
\end{itemize}

\section{Boolean Algebra}

\subsection{Definition}
\begin{definition}
A \textbf{Boolean algebra} is a set $B$ with operations $\land$ (AND), $\lor$ (OR), and $\neg$ (NOT) satisfying certain axioms.
\end{definition}

\subsection{Boolean Identities}
\begin{itemize}
    \item \textbf{Identity Laws}: $x \land 1 = x$, $x \lor 0 = x$
    \item \textbf{Domination Laws}: $x \land 0 = 0$, $x \lor 1 = 1$
    \item \textbf{Idempotent Laws}: $x \land x = x$, $x \lor x = x$
    \item \textbf{Double Complement}: $\neg(\neg x) = x$
    \item \textbf{Commutative Laws}: $x \land y = y \land x$, $x \lor y = y \lor x$
    \item \textbf{Associative Laws}: $(x \land y) \land z = x \land (y \land z)$
    \item \textbf{Distributive Laws}: $x \land (y \lor z) = (x \land y) \lor (x \land z)$
    \item \textbf{De Morgan's Laws}: $\neg(x \land y) = \neg x \lor \neg y$, $\neg(x \lor y) = \neg x \land \neg y$
\end{itemize}

\section{Algorithms and Complexity}

\subsection{Algorithm Analysis}
\begin{itemize}
    \item \textbf{Time Complexity}: How running time grows with input size
    \item \textbf{Space Complexity}: How memory usage grows with input size
\end{itemize}

\subsection{Big-O Notation}
\begin{definition}
$f(n) = O(g(n))$ if there exist constants $c$ and $n_0$ such that $f(n) \leq c \cdot g(n)$ for all $n \geq n_0$.
\end{definition}

\subsection{Common Complexity Classes}
\begin{itemize}
    \item \textbf{Constant}: $O(1)$
    \item \textbf{Logarithmic}: $O(\log n)$
    \item \textbf{Linear}: $O(n)$
    \item \textbf{Linearithmic}: $O(n \log n)$
    \item \textbf{Quadratic}: $O(n^2)$
    \item \textbf{Exponential}: $O(2^n)$
    \item \textbf{Factorial}: $O(n!)$
\end{itemize}

\section{Probability}

\subsection{Basic Definitions}
\begin{definition}
The \textbf{sample space} $S$ is the set of all possible outcomes of an experiment.
\end{definition}

\begin{definition}
An \textbf{event} is a subset of the sample space.
\end{definition}

\subsection{Probability Axioms}
For any event $E$:
\begin{itemize}
    \item $0 \leq P(E) \leq 1$
    \item $P(S) = 1$
    \item For mutually exclusive events: $P(E_1 \cup E_2 \cup \cdots) = P(E_1) + P(E_2) + \cdots$
\end{itemize}

\subsection{Conditional Probability}
\begin{definition}
$$P(A|B) = \frac{P(A \cap B)}{P(B)} \quad \text{where } P(B) > 0$$
\end{definition}

\subsection{Bayes' Theorem}
\begin{theorem}
$$P(A|B) = \frac{P(B|A) \cdot P(A)}{P(B)}$$
\end{theorem}

\subsection{Independent Events}
\begin{definition}
Events $A$ and $B$ are \textbf{independent} if $P(A \cap B) = P(A) \cdot P(B)$.
\end{definition}

\section{Important Theorems and Results}

\subsection{Inclusion-Exclusion Principle}
\begin{theorem}
For finite sets $A_1, A_2, \ldots, A_n$:
$$|A_1 \cup A_2 \cup \cdots \cup A_n| = \sum_{i=1}^{n} |A_i| - \sum_{1 \leq i < j \leq n} |A_i \cap A_j| + \cdots + (-1)^{n+1} |A_1 \cap A_2 \cap \cdots \cap A_n|$$
\end{theorem}

\subsection{Chinese Remainder Theorem}
\begin{theorem}
If $m_1, m_2, \ldots, m_k$ are pairwise relatively prime integers, then the system of congruences:
\begin{align}
x &\equiv a_1 \pmod{m_1} \\
x &\equiv a_2 \pmod{m_2} \\
&\vdots \\
x &\equiv a_k \pmod{m_k}
\end{align}
has a unique solution modulo $m_1 m_2 \cdots m_k$.
\end{theorem}

\subsection{Fermat's Little Theorem}
\begin{theorem}
If $p$ is prime and $\gcd(a,p) = 1$, then $a^{p-1} \equiv 1 \pmod{p}$.
\end{theorem}

\subsection{Wilson's Theorem}
\begin{theorem}
A positive integer $n > 1$ is prime if and only if $(n-1)! \equiv -1 \pmod{n}$.
\end{theorem}

\end{document}
